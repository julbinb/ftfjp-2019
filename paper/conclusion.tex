We have presented decidable subtyping of nominal types, tuples, and unions, 
that has the advantages of semantic subtyping, 
such as simple set-theoretic reasoning, 
yet can be used in the context of dynamically typed languages.
Namely, we interpret types in terms of type tags, 
which are typical for dynamic languages,
and provide a decidable syntactic subtyping relation that is
equivalent to the subset relation on the interpretations
(aka tag-based semantic subtyping).

We found that the proposed subtyping relation, 
if used for multiple dynamic dispatch, 
would make the semantics of dynamically typed programs unstable
due to an interaction of abstract nominal types and unions.
We expect that a different semantic interpretation of nominal types 
can fix the issue, and would like to further explore the alternative.

In future work, we plan to extend tag-based semantic subtyping 
to top and bottom types, 
and also invariant type constructors, e.g. \jltype{Ref}:
\[
\begin{array}{rcl}
\ty \in \Type   &::=& \ldots \Alt \tyref[\ty]\\
\vty \in \VType &::=& \ldots \Alt \tyref[\ty]
\end{array}
\]
As usual for invariant constructors, 
we would like to consider types such as $\tyref[\tyint]$
and $\tyref[\tyunion{\tyint}{\tyint}]$ to be equivalent
because $\tyint$ and $\tyunion{\tyint}{\tyint}$ are equivalent.
However, a naive interpretation of invariant types below
is not well defined:
\[
\interpty{\tyref[\ty]} = 
\{ \tyref[\ty'] \Alt \vty \in \interpty{\ty} \iff \vty \in \interpty{\ty'} \}.
\]
Our plan is to introduce an indexed interpretation,
\[
\interpty{\tyref[\ty]}_{k+1} = \{ \tyref[\ty'] 
    \Alt \vty \in \interpty{\ty}_k \iff \vty \in \interpty{\ty'}_k \},
\]
and define semantic subtyping as:
\[
\bjtruesemsub{\ty_1}{\ty_2} \quad \defsign \quad
\forall k.\ (\interpty{\ty_1}_k \subseteq \interpty{\ty_2}_k).
\]

%where abstract nominal types are interpreted via type tags corresponding 
%to their declared concrete subtypes.