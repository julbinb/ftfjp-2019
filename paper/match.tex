Let us unroll the definition of subset relation\footnote{
$X \subseteq Y \quad \defsign \quad \forall x.\ (x \in X \implies x \in Y)$.}
in the definition of semantic subtyping:
\begin{equation}\label{eq:truesemsub-def}
\bjtruesemsub{\ty_1}{\ty_2} \quad \defsign \quad
\forall \vty.\ (\vty \in \interpty{\ty_1} \implies \vty \in \interpty{\ty_2}).
\end{equation}
Its main ingredient is in-the-interpretation relation $\vty \in \interpty{\ty}$.
We can define an inductive syntax-directed relation
equivalent to $\vty \in \interpty{\ty}$.
We call this relation \defemph{matching relation} 
and denote by $\bjmtch{\vty}{\ty}$, 
read ``value type \vty matches \ty''.
The formal definition is given in~\figref{fig:bjsem-match}.
The matching and in-the-interpretation relations are equivalent:
\begin{equation}\label{eq:match-eq-in-the-interpty}
\forall \vty.\ (\vty \in \interpty{\ty} \ \iff \ \ \bjmtch{\vty}{\ty}).
\end{equation}
The proof is done by induction on \ty (\appref{app:match}).

\begin{figure}
	\begin{mathpar}
		\inferrule*[right=MT-CName]
		{ }
		{ \bjmtch{\cname}{\cname} }		
		\\
		
		\inferrule*[right=\footnotesize{MT-IntReal}]
		{ }
		{ \bjmtch{\tyint}{\tyreal} }
		
		\inferrule*[right=\footnotesize{MT-FltReal}]
		{ }
		{ \bjmtch{\tyflt}{\tyreal} }
		\\
		
		\inferrule*[right=MT-IntNum]
		{ }
		{ \bjmtch{\tyint}{\tynum} }
		
		\inferrule*[right=MT-FltNum]
		{ }
		{ \bjmtch{\tyflt}{\tynum} }
		
		\inferrule*[right=MT-CmplxNum]
		{ }
		{ \bjmtch{\tycmplx}{\tynum} }
		\\
		
		\inferrule*[right=MT-Pair]
		{ \bjmtch{\vty_1}{\ty_1} \\ \bjmtch{\vty_2}{\ty_2} }
		{ \bjmtch{\typair{\vty_1}{\vty_2}}{\typair{\ty_1}{\ty_2}} }
		\\
		
		\inferrule*[right=MT-Union1]
		{ \bjmtch{\vty}{\ty_1}  }
		{ \bjmtch{\vty}{\tyunion{\ty_1}{\ty_2}} }
		
		\inferrule*[right=MT-Union2]
		{ \bjmtch{\vty}{\ty_2}  }
		{ \bjmtch{\vty}{\tyunion{\ty_1}{\ty_2}} }
	\end{mathpar}
	\caption{Matching Relation in \BetaJulia}
	\label{fig:bjsem-match}
\end{figure}

Using the matching relation, we define a \emph{syntactic model} of
semantic subtyping as follows:
\begin{equation}\label{eq:semsub-def}
\bjsemsub{\ty_1}{\ty_2} \quad \defsign \quad
\forall \vty.\ (\bjmtch{\vty}{\ty_1} \implies\ \bjmtch{\vty}{\ty_1}).
\end{equation}
By~\eqref{eq:match-eq-in-the-interpty}, we have that $\bjsemsub{\ty_1}{\ty_2}$
is equivalent to our definition of semantic subtyping for \BetaJulia:
\begin{equation}
\bjsemsub{\ty_1}{\ty_2} \quad \iff \quad \bjtruesemsub{\ty_1}{\ty_2}.
\end{equation}
We will later use the relation $\bjsemsub{\ty_1}{\ty_2}$
to prove that an inductive definition of subtyping coincides
with the semantic interpretation.

While the semantic interpretation does provide a useful intuition for subtyping,
neither of the definitions~\eqref{eq:truesemsub-def}, \eqref{eq:semsub-def} 
can be implemented directly because of the quantification $\forall \vty$.
Therefore, we need to provide another definition of subtyping, which would
be equivalent to the semantic one, but straightforward to implement.

We do this in two steps. 
First, we give an inductive definition of subtyping that is 
handy to reason about and prove equivalent to the semantic definition.
We call this subtyping \emph{declarative}.
Second, we provide a \emph{reductive}, syntax-directed inductive definition that is equivalent to the declarative subtyping 
(and, hence, the semantic one as well)
and also decidable, that is for any two types $\ty_1$ and $\ty_2$
we can prove that either $\ty_1$ is a subtype of $\ty_2$ or it is not.
