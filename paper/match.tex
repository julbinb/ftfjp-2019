Let us unroll the definition of semantic subtyping\footnote{In set theory,
the subset relation
$X \subseteq Y \quad \defsign \quad \forall x.\ (x \in X \implies x \in Y)$.}:
\begin{equation}\label{eq:truesemsub-def}
\bjtruesemsub{\ty_1}{\ty_2} \quad \defsign \quad
\forall \vty.\ (\vty \in \interpty{\ty_1} \implies \vty \in \interpty{\ty_2}).
\end{equation}
Its main ingredient is the relation %member-of-interpretation relation 
$\vty \in \interpty{\ty}$,
which has a syntactic counterpart~---
an equivalent\footnote{It is easy to show by
induction on \ty that the matching relation is sound and complete 
with respect to the member-of-interpretation relation, i.e.
$\forall \vty.\ (\bjmtch{\vty}{\ty} \ \iff \ \ \vty \in \interpty{\ty})$.} 
inductive, syntax-directed relation $\bjmtch{\vty}{\ty}$
defined in~\figref{fig:bjsem-match}.
We call the latter 
\defemph{matching relation}, read ``tag~\vty matches type~\ty''.

\begin{figure}
	\begin{mathpar}
		\inferrule*[right=MT-CName]
		{ }
		{ \bjmtch{\cname}{\cname} }		
		\\
		
		\inferrule[MT-IntReal]
		{ }
		{ \bjmtch{\tyint}{\tyreal} }
		
		\inferrule[MT-FltReal]
		{ }
		{ \bjmtch{\tyflt}{\tyreal} }
		\\
		
		\inferrule[MT-IntNum]
		{ }
		{ \bjmtch{\tyint}{\tynum} }
		
		\inferrule[MT-FltNum]
		{ }
		{ \bjmtch{\tyflt}{\tynum} }
		
		\inferrule[MT-CmplxNum]
		{ }
		{ \bjmtch{\tycmplx}{\tynum} }
		\\
		
		\inferrule*[right=MT-Pair]
		{ \bjmtch{\vty_1}{\ty_1} \\ \bjmtch{\vty_2}{\ty_2} }
		{ \bjmtch{\typair{\vty_1}{\vty_2}}{\typair{\ty_1}{\ty_2}} }
		\\
		
		\inferrule*[right=MT-Union1]
		{ \bjmtch{\vty}{\ty_1}  }
		{ \bjmtch{\vty}{\tyunion{\ty_1}{\ty_2}} }
		
		\inferrule*[right=MT-Union2]
		{ \bjmtch{\vty}{\ty_2}  }
		{ \bjmtch{\vty}{\tyunion{\ty_1}{\ty_2}} }
	\end{mathpar}
	\caption{Matching Relation in \BetaJulia}
	\label{fig:bjsem-match}
\end{figure}

Using the matching relation, we define a syntactic model of
semantic subtyping $\bjsemsub{\ty_1}{\ty_2}$ as follows:
\begin{equation}\label{eq:semsub-def}
\bjsemsub{\ty_1}{\ty_2} \quad \defsign \quad
\forall \vty.\ (\bjmtch{\vty}{\ty_1} \implies\ \bjmtch{\vty}{\ty_2}).
\end{equation}
Because $\vty \in \interpty{\ty}$ and $\bjmtch{\ty_1}{\ty_2}$
are equivalent, definitions~\eqref{eq:truesemsub-def} and~\eqref{eq:semsub-def}
are also equivalent:
%, i.e. relation $\bjsemsub{\ty_1}{\ty_2}$ 
%is sound ($\implies$)
%and complete (${\impliedby}$) with respect to $\bjtruesemsub{\ty_1}{\ty_2}$:
\begin{equation}
\bjsemsub{\ty_1}{\ty_2} \quad \iff \quad \bjtruesemsub{\ty_1}{\ty_2}.
\end{equation}

%We will later use the relation $\bjsemsub{\ty_1}{\ty_2}$
%to prove that an inductive definition of subtyping coincides
%with the semantic interpretation.

While the semantic approach does provide a useful intuition for subtyping,
neither of the definitions~\eqref{eq:truesemsub-def}, \eqref{eq:semsub-def} 
can be directly computed because of the quantification $\forall \vty$.
Therefore, we need to provide another, \emph{syntactic} 
definition of subtyping, which would
be equivalent to the semantic one but also straightforward to implement.

We do this in two steps. 
First, we give an inductive definition of subtyping, 
called \emph{declarative}, that is handy to reason about. 
We prove it equivalent to the semantic subtyping.
Second, we provide a \emph{reductive}, syntax-directed definition that is equivalent to the declarative subtyping 
(and, hence, the semantic one as well)
and also decidable, that is for any two types $\ty_1$ and $\ty_2$
we can prove that $\ty_1$ either is a subtype of $\ty_2$ or is not.
