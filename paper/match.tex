Let us unroll the definition of semantic subtyping\footnote{In set theory,
the subset relation
$X \subseteq Y \quad \defsign \quad \forall x.\ (x \in X \implies x \in Y)$.}:
\begin{equation}\label{eq:truesemsub-def}
\bjtruesemsub{\ty_1}{\ty_2} \quad \defsign \quad
\forall \vty.\ (\vty \in \interpty{\ty_1} \implies \vty \in \interpty{\ty_2}).
\end{equation}
Its main ingredient is the relation %member-of-interpretation relation 
$\vty \in \interpty{\ty}$,
which has a syntactic counterpart~---
an equivalent\footnote{It is easy to show by
induction on \ty that the matching relation is sound and complete 
with respect to the member-of-interpretation relation, i.e.
$\forall \vty.\ (\bjmtch{\vty}{\ty} \ \iff \ \ \vty \in \interpty{\ty})$.} 
inductive, syntax-directed relation $\bjmtch{\vty}{\ty}$
defined in~\figref{fig:bjsem-match}.
We call the latter 
a \defemph{matching relation}, read ``tag~\vty matches type~\ty''.

\begin{figure}
	\begin{mathpar}
		\inferrule*[right=MT-CName]
		{ }
		{ \bjmtch{\cname}{\cname} }		
		\\
		
		\inferrule[MT-IntReal]
		{ }
		{ \bjmtch{\tyint}{\tyreal} }
		
		\inferrule[MT-FltReal]
		{ }
		{ \bjmtch{\tyflt}{\tyreal} }
		\\
		
		\inferrule[MT-IntNum]
		{ }
		{ \bjmtch{\tyint}{\tynum} }
		
		\inferrule[MT-FltNum]
		{ }
		{ \bjmtch{\tyflt}{\tynum} }
		
		\inferrule[MT-CmplxNum]
		{ }
		{ \bjmtch{\tycmplx}{\tynum} }
		\\
		
		\inferrule*[right=MT-Pair]
		{ \bjmtch{\vty_1}{\ty_1} \\ \bjmtch{\vty_2}{\ty_2} }
		{ \bjmtch{\typair{\vty_1}{\vty_2}}{\typair{\ty_1}{\ty_2}} }
		\\
		
		\inferrule*[right=MT-Union1]
		{ \bjmtch{\vty}{\ty_1}  }
		{ \bjmtch{\vty}{\tyunion{\ty_1}{\ty_2}} }
		
		\inferrule*[right=MT-Union2]
		{ \bjmtch{\vty}{\ty_2}  }
		{ \bjmtch{\vty}{\tyunion{\ty_1}{\ty_2}} }
	\end{mathpar}
	\caption{Matching relation in \BetaJulia}
	\label{fig:bjsem-match}
\end{figure}

Using the matching relation, we define a syntactic model of
semantic subtyping $\bjsemsub{\ty_1}{\ty_2}$ as follows:
\begin{equation}\label{eq:semsub-def}
\bjsemsub{\ty_1}{\ty_2} \quad \defsign \quad
\forall \vty.\ (\bjmtch{\vty}{\ty_1} \implies\ \bjmtch{\vty}{\ty_2}).
\end{equation}
Because $\vty \in \interpty{\ty}$ and $\bjmtch{\vty}{\ty}$
are equivalent, definitions~\eqref{eq:truesemsub-def} and~\eqref{eq:semsub-def}
are also equivalent:
%, i.e. relation $\bjsemsub{\ty_1}{\ty_2}$ 
%is sound ($\implies$)
%and complete (${\impliedby}$) with respect to $\bjtruesemsub{\ty_1}{\ty_2}$:
\begin{equation}
\bjsemsub{\ty_1}{\ty_2} \quad \iff \quad \bjtruesemsub{\ty_1}{\ty_2}.
\end{equation}
We will use the syntactic model~\eqref{eq:semsub-def}
in~\secref{sec:declsub-correct}
to prove that alternative, syntactic definitions of subtyping
coincide with the semantic interpretation.
%to prove that an inductive definition of subtyping coincides
%with the semantic interpretation.

