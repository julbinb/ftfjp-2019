Subtyping \ldots (roughly what it is; is used both in static and dynamic
languages).

Two flavors of subtyping: syntactic and semantic. 

Semantic subtyping enables simple set-theoretic reasoning about types
in terms of values they represent.
Because of that, it is especially appealing to the users of a language.
However, semantic subtyping might be problematic for the language designers:
defining the interpretation of types and developing a corresponding
sound and complete subtyping algorithm is usually quite tricky.

Prior work on semantic subtyping was mostly focused on rather complex languages
with structural types (XML-CDuce or Object-Oriented Ancona).
For example, \ldots. Often care about static typing.

In this work, we explore the applicability of semantic subtyping in the context
of a \emph{dynamic language with nominal types} 
where all \emph{values are type tagged}. 
This allows us to considerably simplify the traditional notion 
of semantic subtyping.
Namely, we interpret a type as a set of type tags, 
with a tag faithfully characterizing a set of values.

We draw our inspiration from the dynamically typed language Julia,
which employs nominal types in a peculiar fashion. For performance reasons, 
only leaf types in the hierarchy of nominal types
can be used to construct values.
As an example, consider two types, \jltype{Int64} and \jltype{Signed}:
\jltype{Int64} is a leaf type, 
and \jltype{Signed} is its declared supertype.
Any 64-bit machine integer value is tagged with the type \jltype{Int64}
and considered to be of type \jltype{Int64}.
Such a value is also considered to be of type \jltype{Signed},
but \emph{no values are tagged} with \jltype{Signed}.
Thus, we can give the semantic interpretation of \jltype{Signed}
solely in terms of its leaf subtypes:
\[
\interpty{\tysigned} = \{\tyintsf, \tyinttt, \tyintst, \ldots\}.
\]

Our contributions are as follows:
\begin{itemize}
  \item semantic interpretation of nominal types for a Julia-like 
    dynamic language (\secref{todo});
  \item simple syntactic model of semantic subtyping for nominal types,
    covariant pairs, and untagged unions (\secref{todo});
  \item decidable reductive subtyping provably equivalent to \ldots (\secref{todo});
  \item discussion of the applicability of semantic subtyping to Julia (\secref{todo}).
  \item Coq!
\end{itemize}

%Besides nominal types, we take covariant pairs and untagged unions.

%That is, any value of type \jltype{Signed} 
%is tagged with one of its leaf subtypes,
%and we can characterize the set of values represented by the type 
%solely in terms of the leaf types.
%Because of that, the type \jltype{Signed} can be interpreted in terms
%of its leaf subtypes.

%The designers of Julia, in turn, were inspired
%by the work on semantic subtyping~\citemock.
%In Julia, all values have type tags that never change at run-time.
%Therefore, tags faithfully characterize sets of values.
%Julia uses subtyping extensively for multiple dynamic dispatch. 






