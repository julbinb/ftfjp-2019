Subtyping \ldots (roughly what it is; is used both in static and dynamic
languages).

Two flavors of subtyping: syntactic and semantic. 

Semantic subtyping enables simple set-theoretic reasoning about types
in terms of values they represent.
Because of that, it is especially appealing to the users of a language.
However, semantic subtyping might be problematic for the language designers:
defining the interpretation of types and developing a corresponding
sound and complete subtyping algorithm is usually quite tricky.

Prior work on semantic subtyping was mostly focused on rather complex languages
with structural types (XML-CDuce or Object-Oriented Ancona).
For example, \ldots. Often care about typing.

In this work, we explore the applicability of semantic subtyping in the context
of a dynamic language with nominal types. We draw our inspiration from
the dynamically typed language Julia, which uses subtyping extensively
for multiple dynamic dispatch. 
%The designers of Julia, in turn, were inspired
%by the work on semantic subtyping~\citemock.
In Julia, all values have type tags that never change at run-time.
This allows us to considerably simplify the traditional notion 
of semantic subtyping.
%Therefore, tags faithfully characterize sets of values.
Namely, we interpret a type as a set of type tags, 
with a tag faithfully characterizing a set of values.





