The declarative definition is not syntax-directed because it involves
the \RD{Trans} rule, which requires ``coming up'' with a middle type $\ty_2$.
For instance, in order to show
\[\bjsub{\typair{\tystr}{\tyreal}}
{(\typair{\tystr}{\tyint}) \cup (\typair{\tystr}{\tystr}) 
	\cup (\typair{\tystr}{\tyflt})},\]
we need to apply transitivity several times, in particular, 
with the middle type $\typair{\tystr}{(\tyunion{\tyint}{\tyflt})}$.

\begin{figure}
	\begin{mathpar}
		\colorbox{light-gray}{$
		\inferrule*[right=SR-BaseRefl]
		{ }
		{ \bjsubr{\cname}{\cname} }
		$}
		\\
		
		\inferrule[SR-IntReal]
		{ }
		{ \bjsubr{\tyint}{\tyreal} }
		
		\inferrule[SR-FltReal]
		{ }
		{ \bjsubr{\tyflt}{\tyreal} }
		\\
	
		\inferrule[SR-CmplxNum]
		{ }
		{ \bjsubr{\tycmplx}{\tynum} }
		
		\colorbox{light-gray}{$
		\inferrule[SR-IntNum]
		{ }
		{ \bjsubr{\tyint}{\tynum} }
		$}
		
		\colorbox{light-gray}{$
		\inferrule[SR-FltNum]
		{ }
		{ \bjsubr{\tyflt}{\tynum} }
		$}
		\\
		
		\inferrule*[right=SR-Pair]
		{ \bjsubr{\ty_1}{\ty'_1} \\ \bjsubr{\ty_2}{\ty'_2} }
		{ \bjsubr{\typair{\ty_1}{\ty_2}}{\typair{\ty'_1}{\ty'_2}} }
		\\
		
		\inferrule*[right=SR-UnionL]
		{ \bjsubr{\ty_1}{\ty'} \\ \bjsubr{\ty_2}{\ty'} }
		{ \bjsubr{\tyunion{\ty_1}{\ty_2}}{\ty'} }
		\\
		
		\colorbox{light-gray}{$
		\inferrule[SR-UnionR1]
		{ \bjsubr{\ty}{\ty'_1} }
		{ \bjsubr{\ty}{\tyunion{\ty'_1}{\ty'_2}} }
		$}
		
		\colorbox{light-gray}{$
		\inferrule[SR-UnionR2]
		{ \bjsubr{\ty}{\ty'_2} }
		{ \bjsubr{\ty}{\tyunion{\ty'_1}{\ty'_2}} }
		$}
		\\
		
		\colorbox{light-gray}{$
		\inferrule*[right=SR-NF]
		{ \bjsubr{\NF(\ty)}{\ty'} }
		{ \bjsubr{\ty}{\ty'} }
		$}
	\end{mathpar}
	\caption{Reductive subtyping for \BetaJulia}
	\label{fig:bjsem-red-sub}
\end{figure}

\figref{fig:bjsem-red-sub} presents the syntax-directed reductive 
definition of subtyping, which can be easily turned into 
a subtyping algorithm.\footnote{The definition is not deterministic, though.
For example, there are two ways to derive 
$\bjsubr{\typair{\tystr}{(\tyunion{\tyint}{\tyflt})}}
{\typair{\tystr}{(\tyunion{\tyint}{\tyflt})}}$: either by immediately 
applying \RR{Pair}, or by first normalizing the left-hand side with \RR{NF}.}
Some of the inductive rules are similar to the declarative definition,
e.g. subtyping of pairs (\RR{Pair}) and the left-hand side union (\RR{UnionL}).
The differing rules are \colorbox{light-gray}{highlighted}.
Instead of the general reflexivity rule \RD{Refl}, the reductive definition
has reflexivity only for concrete nominal types (\RR{BaseRefl}).
General transitivity is also gone; instead, it gets incorporated into
subtyping of nominal types (\RR{IntNum}, \RR{FltNum})
and the right-hand side union (\RR{UnionR1}, \RR{UnionR2}),
as well as the rule for using the normal form (\RR{NF}).
Note that the last rule also takes care of distributivity.

With the reductive subtyping, the example above can be derived as follows:
\begin{mathpar}\small
\inferrule*[right=]
{ \inferrule*[right=]
  { \bjsub{\typair{\tystr}{\tyint}}{(\typair{\tystr}{\tyint}) \ldots} \\
    \bjsub{\typair{\tystr}{\tyflt}}{\ldots (\typair{\tystr}{\tyflt})}  }	
  { \bjsub{\tyunion{(\typair{\tystr}{\tyint})}{(\typair{\tystr}{\tyflt})}}
	{(\typair{\tystr}{\tyint}) \cup \ldots \cup (\typair{\tystr}{\tyflt})} } }
{ \bjsub{\typair{\tystr}{\tyreal}}
	{(\typair{\tystr}{\tyint}) \cup (\typair{\tystr}{\tystr}) 
		\cup (\typair{\tystr}{\tyflt})} }.
\end{mathpar}
In the very bottom, we use \RR{NF}, and then \RR{UnionL} in the level above.
To complete the top of derivation, we would also need to use \RR{UnionR*}, 
\RR{Pair}, and \RR{BaseRefl}.

%% -----------------------------------------------------------------------------

\begin{theorem}[Correctness of Reductive Subtyping]\label{thm:redsub-correct}
\[
\forall \ty_1, \ty_2.\ (\bjsubr{\ty_1}{\ty_2} \iff\ \bjsub{\ty_1}{\ty_2})
\]
\end{theorem}

It is relatively easy to show by induction on a derivation of
$\bjsubr{\ty_1}{\ty_2}$ that the reductive subtyping is sound:
for each case we build a corresponding derivation of $\bjsub{\ty_1}{\ty_2}$.
Most of the reductive rules have direct declarative counterparts.
In the case of \RR{*Num} and \RR{UnionR*}, 
we need to additionally use transitivity.
Finally, in the case of \RR{NF}, the induction hypothesis gives us
$\bjsub{\NF(\ty_1)}{\ty_2}$, so we can use 
\lemref{lem:declsub-nf} and \RD{Trans} to derive $\bjsub{\ty_1}{\ty_2}$.

The challenging part of the proof is to show completeness. 
For this, we need to prove that 
the reductive definition is \emph{reflexive}, \emph{transitive},
and \emph{distributive}.
To prove transitivity and distributivity, 
we need several auxiliary statements:
\begin{enumerate}
  \item $\bjsubr{\ty}{\ty'}\ \implies\ \bjsubr{\NF(\ty)}{\ty'}$,
  \item $\bjsubr{\NF(\ty_1)}{\NF(\ty_2)}\;\;\ \land\;\; \bjsubr{\NF(\ty_2)}{\ty_3}$\\
     $\implies \bjsubr{\NF(\ty_1)}{\ty_3}$,
  \item $\bjsubr{\NF(\ty)}{\NF(\ty')}\ \implies\ \bjsubr{\ty}{\ty'}$.
\end{enumerate}
Having all the facts, we can prove completeness by induction
on a derivation of $\bjsub{\NF(\ty_1)}{\ty_2}$.
For details, the reader can refer to the full Coq-proof.

\begin{theorem}[Decidability of Reductive Subtyping]\label{thm:redsub-decidable}
\[
\forall \ty_1, \ty_2.\ 
(\bjsubr{\ty_1}{\ty_2}\quad \lor\quad \lnot \bjsub{\ty_1}{\ty_2})
\]
\end{theorem}

By induction on a derivation of $\InNF(\ty_1)$, we can show
that the reductive subtyping is decidable for $\ty_1 | \InNF(\ty_1)$.
In this case, $\ty_1$ is either a value type or a union of normalized types.
If it is a value type, subtyping is equivalent to matching,
and matching is decidable.
If $\ty_1$ is a union $\tyunion{\ty_a}{\ty_b}$, 
we use induction hypothesis for components and the fact that
subtyping union on the left-hand side
$\bjsubr{\tyunion{\ty_a}{\ty_b}}{\ty'}$ implies $\bjsubr{\ty_a}{\ty'}$
and $\bjsubr{\ty_b}{\ty'}$.
Since $\InNF(\NF(\ty_1))$, we can use the decidability 
of $\bjsubr{\NF(\ty_1)}{\ty_2}$ to prove decidability
of $\bjsubr{\ty_1}{\ty_2}$.