\subsection{Reductive Subtyping}\label{sec:redsub}
%% -----------------------------------------------------------------------------

The declarative definition is neither syntax-directed nor analytic
and cannot be directly turned into a subtyping algorithm.
For one, the transitivity rule \RD{Trans} 
overlaps with any other rule in the system
and also requires ``coming up'' with an intermediate type $\ty_2$
to conclude $\bjsub{\ty_1}{\ty_3}$.
For instance, to derive %in order
\[\bjsub{\typair{\tystr}{\tyreal}}
{(\typair{\tystr}{\tyint}) \cup (\typair{\tystr}{\tystr}) 
	\cup (\typair{\tystr}{\tyflt})},\]
we need to apply transitivity several times, in particular, 
with the intermediate type $\typair{\tystr}{(\tyunion{\tyint}{\tyflt})}$.
Another source of overlap is the reflexivity and distributivity rules.
%Another issue is that the rules \RD{Refl}, \RD{UnionR*}, and \RD{Distr*}
%require deep structural comparison of the left- and right-hand side types.
%For example, in order for \RD{UnionR1} to be applicable to some 
%$\bjsub{\ty_a}{\ty_b}$, not only $\ty_b$ must be a union, 
%but also its left component should be equal to $\ty_a$.

\begin{figure}
%	\hfill \fbox{$\bjsubr{\ty}{\ty'}$}
	\begin{mathpar}
		\fbox{$\bjsubr{\ty}{\ty'}$}\\
		
		\colorbox{light-gray}{$
		\inferrule*[right=SR-BaseRefl]
		{ }
		{ \bjsubr{\cname}{\cname} }
		$}
		\\
		
		\inferrule[SR-IntReal]
		{ }
		{ \bjsubr{\tyint}{\tyreal} }
		
		\inferrule[SR-FltReal]
		{ }
		{ \bjsubr{\tyflt}{\tyreal} }
		\\
	
		\inferrule[SR-CmplxNum]
		{ }
		{ \bjsubr{\tycmplx}{\tynum} }
		
		\colorbox{light-gray}{$
		\inferrule[SR-IntNum]
		{ }
		{ \bjsubr{\tyint}{\tynum} }
		$}
		
		\colorbox{light-gray}{$
		\inferrule[SR-FltNum]
		{ }
		{ \bjsubr{\tyflt}{\tynum} }
		$}
		\\
		
		\inferrule*[right=SR-Pair]
		{ \bjsubr{\ty_1}{\ty'_1} \\ \bjsubr{\ty_2}{\ty'_2} }
		{ \bjsubr{\typair{\ty_1}{\ty_2}}{\typair{\ty'_1}{\ty'_2}} }
		\\
		
		\inferrule*[right=SR-UnionL]
		{ \bjsubr{\ty_1}{\ty'} \\ \bjsubr{\ty_2}{\ty'} }
		{ \bjsubr{\tyunion{\ty_1}{\ty_2}}{\ty'} }
		\\
		
		\colorbox{light-gray}{$
		\inferrule[SR-UnionR1]
		{ \bjsubr{\ty}{\ty'_1} }
		{ \bjsubr{\ty}{\tyunion{\ty'_1}{\ty'_2}} }
		$}
		
		\colorbox{light-gray}{$
		\inferrule[SR-UnionR2]
		{ \bjsubr{\ty}{\ty'_2} }
		{ \bjsubr{\ty}{\tyunion{\ty'_1}{\ty'_2}} }
		$}
		\\
		
		\colorbox{light-gray}{$
		\inferrule*[right=SR-NF]
		{ \bjsubr{\NF(\ty)}{\ty'} }
		{ \bjsubr{\ty}{\ty'} }
		$}
	\end{mathpar}
	\caption{Reductive subtyping for \BetaJulia}
	\label{fig:bjsem-red-sub}
\end{figure}

%The syntax-directed reductive definition\footnote{The definition
%	%the rules can be easily turned 
%	%into a subtyping algorithm.
%	is not deterministic, though.	For example, there are two ways to derive 
%	$\bjsubr{\typair{\tystr}{(\tyunion{\tyint}{\tyflt})}}
%	{\typair{\tystr}{(\tyunion{\tyint}{\tyflt})}}$: 
%	either by immediately applying \RR{Pair}, 
%	or by first normalizing the left-hand side with \RR{NF}.} of subtyping
%is presented in~\figref{fig:bjsem-red-sub}.
By contrast, the rules of reductive subtyping enable
straightforward bottom up reasoning;
the rules are presented in~\figref{fig:bjsem-red-sub}.
The reductive definition lacks the most problematic rules
of declarative subtyping, 
i.e. general reflexivity, transitivity, and distributivity.
Some of the inductive rules have the exact declarative counterparts,
e.g. subtyping of pairs (\RR{Pair}) or
subtyping of a union on the left (\RR{UnionL}).
%for instance, the rule for subtyping of pairs (\RR{Pair}) 
%or the rule for a union on the left (\RR{UnionL}).

The differing rules are \colorbox{light-gray}{highlighted}.
The explicit reflexivity rule \RR{BaseRefl} now only works with 
concrete nominal types, 
but this already makes the reductive definition reflexive.
%for the reductive definition to be reflexive.
The definition also has to be transitive,
so several rules are added or modified to enable derivations
that used to rely on transitivity in the declarative definition.
These include subtyping of nominal types (\RR{IntNum}, \RR{FltNum}),
subtyping of a union on the right (\RR{UnionR1}, \RR{UnionR2}),
and normalization (\RR{NF}).

The last rule of the definition, \RR{NF}, is the most important,
as it covers all useful interactions of transitivity and distributivity 
that are possible in the declarative definition.
The rule rewrites type \ty into its \defemph{normal form} $\NF(\ty)$
before applying other subtyping rules.
%This covers all useful applications of transitivity and distributivity 
%that are possible in the declarative definition.
Any normalized type has the form $\vty_1 \cup \vty_2 \cup \ldots \cup \vty_n$,
i.e. a union of value types
(we omit parenthesis because union is associative).
The normalization function $\NF$ is presented in~\figref{fig:bjsem-calc-nf}
(the auxiliary function $\unprs$ 
can be found in~\figref{fig:bjsem-calc-nf-full}, \appref{app:nf}).
It produces a type in \emph{disjunctive normal form}
by replacing an abstract nominal type 
with the union of all its concrete subtypes, 
and a pair of unions with the union of pairs of value types
(each of this pairs is itself a value type),
for instance:
\[
\NF(\typair{\tystr}{(\tyunion{\tyint}{\tyflt})}) =
\tyunion{(\typair{\tystr}{\tyint})}{(\typair{\tystr}{\tyflt})}.
\]
As shown in~\secref{sec:declsub-correct}, a type and its normal form are
equivalent according to the declarative definition.
This property is essential for the reductive subtyping 
being equivalent to the declarative one.
%an essential property
%for reductive subtyping being equivalent to declarative one.

\begin{figure}
  \[
	\begin{array}{rcl}
	\NF: \Type &\rightarrow& \Type \\
	\NF(\cname) &=& \cname \\
	\colorbox{light-gray}{\NF(\tyreal)} &=&
	\colorbox{light-gray}{\tyunion{\tyint}{\tyflt}} \\
	\colorbox{light-gray}{\NF(\tynum)} &=&
	\colorbox{light-gray}{\tyunion{\tyunion{\tyint}{\tyflt}}{\tycmplx}} \\
	\NF(\typair{\ty_1}{\ty_2}) &=& \unprs(\NF(\ty_1), \, \NF(\ty_2))	\\
	\NF(\tyunion{\ty_1}{\ty_2}) &=& \tyunion{\NF(\ty_1)}{\NF(\ty_2)} \\
	\end{array}
  \]
	\caption{Computing normal form of \BetaJulia types}
	\label{fig:bjsem-calc-nf}
\end{figure}

\paragraph{Subtyping Algorithm.}
The reductive rules are analytic,
and if a derivation of $\bjsub{\ty}{\ty'}$ exists,
it can always be found by the following algorithm.
\begin{enumerate}%[1)]
  \item Use the normalization rule \RR{NF} once (normalize $\ty$);
  \item Use all the other rules to derive 
    $\bjsub{\NF(\ty)}{\ty'}$ in the standard manner, bottom up;
    except for an overlap between \RR{UnionR1} and \RR{UnionR2},
    these rules are syntax-directed.
\end{enumerate}
%(1) apply the normalization rule \RR{NF} once, i.e. normalize $\ty$;
%(2) use all the other (syntax-directed) rules to build the derivation of
%$\bjsub{\NF(\ty)}{\ty'}$ in the usual manner, bottom to top.

However, this algorithm does not always produce the shortest derivation.
For instance, for
$\bjsubr{\typair{\tystr}{(\tyunion{\tyint}{\tyflt})}}
	    {\typair{\tystr}{\tyreal}}$,
it produces a derivation with eight applications of the rules, 
whereas the shortest derivation needs only five applications
(see~\appref{app:example-deriv}).
It is possible that in practice, an algorithm that tries the short path first 
and only then resorts to normalization would work better.

The actual Julia implementation uses a clever algorithm 
to check subtyping of tuples and unions 
without having to normalize types~\cite{bib:Chung19}.
The algorithm is equivalent to the normalization-based one discussed above,
but instead of computing the whole normal form, 
it computes only the components of the normalized type, one at a time.

Note that the rules for subtyping of nominal types do not have to be built-in.
Instead of five separate rules, as presented in~\figref{fig:bjsem-red-sub},
we can use a single rule that relies on the relation 
$\bjnomsub{n_1}{n_2}$ ($n_1$ transitively extends $n_2$)
from~\secref{subsec:interp}:
\begin{mathpar}
	\inferrule*[right=SR-Nom]
	{ \bjnomsub{n_1}{n_2} }
	{ \bjsub{n_1}{n_2} }.
\end{mathpar}
Then, for any $n_1$ and $n_2$, the relation $\bjnomsub{n_1}{n_2}$ 
can be checked algorithmically, using the nominal hierarchy $\NomH$.


% for using the normal form (\RR{NF}).
%Note that the last rule also takes care of distributivity.

%With the reductive subtyping, the example above can be derived as follows:
%\begin{mathpar}\small
%\inferrule*[right=]
%{ \inferrule*[right=]
%  { \bjsub{\typair{\tystr}{\tyint}}{(\typair{\tystr}{\tyint}) \ldots} \\
%    \bjsub{\typair{\tystr}{\tyflt}}{\ldots (\typair{\tystr}{\tyflt})}  }	
%  { \bjsub{\tyunion{(\typair{\tystr}{\tyint})}{(\typair{\tystr}{\tyflt})}}
%	{(\typair{\tystr}{\tyint}) \cup \ldots \cup (\typair{\tystr}{\tyflt})} } }
%{ \bjsub{\typair{\tystr}{\tyreal}}
%	{(\typair{\tystr}{\tyint}) \cup (\typair{\tystr}{\tystr}) 
%		\cup (\typair{\tystr}{\tyflt})} }.
%\end{mathpar}
%In the very bottom, we use \RR{NF}, and then \RR{UnionL} in the level above.
%To complete the top of derivation, we would also need to use \RR{UnionR*}, 
%\RR{Pair}, and \RR{BaseRefl}.

