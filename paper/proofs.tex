\subsection{Correctness of Declarative Subtyping}\label{sec:declsub-correct}
%% -----------------------------------------------------------------------------

In order to show that the declarative definition of subtyping 
is equivalent to the semantic definition,
we need to prove that the former is sound and complete with respect
to the latter, that is:
\begin{equation}\label{eq:declsub-correct-truesemsub}
\forall \ty_1, \ty_2.\ (\bjsub{\ty_1}{\ty_2} \iff \bjtruesemsub{\ty_1}{\ty_2}).
\end{equation}
We find that instead of directly working 
with the semantic subtyping relation $\bjtruesemsub{\ty_1}{\ty_2} \defsign
\interpty{\ty_1} \subseteq \interpty{\ty_2}$~\eqref{eq:truesemsub-def},
it is more convenient to use the following (equivalent) relation:
\begin{equation}\label{eq:semsub-def}
\bjsemsub{\ty_1}{\ty_2} \quad \defsign \quad
\forall \vty.\ (\bjmtch{\vty}{\ty_1} \implies\ \bjmtch{\vty}{\ty_2}).
\end{equation}
The relation $\bjmtch{\vty}{\ty}$ (defined in~\figref{fig:bjsem-match}), 
read ``tag~\vty matches type~\ty'', 
we call \defemph{matching relation}.
It is easy to show by induction on \ty that the matching relation
is equivalent to the belongs-to relation $\vty \in \interpty{\ty}$.
And the latter relation is used in the definition of semantic subtyping
(we replace set inclusion $X \subseteq Y$ with its definition 
$\forall x.\ (x \in X \implies x \in Y)$):
\begin{equation}\label{eq:truesemsub-def-new}
\bjtruesemsub{\ty_1}{\ty_2} \;\defsign \;
%\interpty{\ty_1} \subseteq \interpty{\ty_2} \; \defsign \;
\forall \vty.\ (\vty \in \interpty{\ty_1} \implies \vty \in \interpty{\ty_2}).
\end{equation}
The definitions~\eqref{eq:truesemsub-def-new} and~\eqref{eq:semsub-def}
are equivalent because the relations 
$\vty \in \interpty{\ty}$ and $\bjmtch{\vty}{\ty}$ are equivalent.
Therefore, instead of proving~\eqref{eq:declsub-correct-truesemsub},
we are going to prove that declarative subtyping $\bjsub{\ty_1}{\ty_2}$
is equivalent to semantic subtyping $\bjsemsub{\ty_1}{\ty_2}$.

\begin{figure}
	\begin{mathpar}
		\inferrule*[right=MT-CName]
		{ }
		{ \bjmtch{\cname}{\cname} }		
		\\
		
		\inferrule[MT-IntReal]
		{ }
		{ \bjmtch{\tyint}{\tyreal} }
		
		\inferrule[MT-FltReal]
		{ }
		{ \bjmtch{\tyflt}{\tyreal} }
		\\
		
		\inferrule[MT-IntNum]
		{ }
		{ \bjmtch{\tyint}{\tynum} }
		
		\inferrule[MT-FltNum]
		{ }
		{ \bjmtch{\tyflt}{\tynum} }
		
		\inferrule[MT-CmplxNum]
		{ }
		{ \bjmtch{\tycmplx}{\tynum} }
		\\
		
		\inferrule*[right=MT-Pair]
		{ \bjmtch{\vty_1}{\ty_1} \\ \bjmtch{\vty_2}{\ty_2} }
		{ \bjmtch{\typair{\vty_1}{\vty_2}}{\typair{\ty_1}{\ty_2}} }
		\\
		
		\inferrule*[right=MT-Union1]
		{ \bjmtch{\vty}{\ty_1}  }
		{ \bjmtch{\vty}{\tyunion{\ty_1}{\ty_2}} }
		
		\inferrule*[right=MT-Union2]
		{ \bjmtch{\vty}{\ty_2}  }
		{ \bjmtch{\vty}{\tyunion{\ty_1}{\ty_2}} }
	\end{mathpar}
	\caption{Matching relation in \BetaJulia}
	\label{fig:bjsem-match}
\end{figure}

\begin{theorem}[Correctness of Declarative Subtyping]\label{thm:declsub-correct}
\[
\forall \ty_1, \ty_2.\ (\bjsub{\ty_1}{\ty_2} \iff\ \bjsemsub{\ty_1}{\ty_2})
\]
\end{theorem}

In order to prove the theorem, we need several auxiliary observations.
First of all,
subtyping a value type coincides with matching:
\begin{equation}\label{eq:mtch-eq-declsub}
\forall \vty, \ty.\ (\bjsub{\vty}{\ty} \iff\ \bjmtch{\vty}{\ty}).
\end{equation}
Having that, it is easy to prove the \emph{soundness} direction
of~\thmref{thm:declsub-correct}.
\begin{lemma}[Soundness of Declarative Subtyping]
\[
\forall \ty_1, \ty_2.\ 
[\bjsub{\ty_1}{\ty_2} \implies\ 
\forall \vty.\ (\bjmtch{\vty}{\ty_1} \implies\ \bjmtch{\vty}{\ty_2})]
\]
\end{lemma}
\noindent
\Proof. We know $\bjmtch{\vty}{\ty_1}$ and $\bjsub{\ty_1}{\ty_2}$.
We need to show that $\bjmtch{\vty}{\ty_2}$.
First, we apply~\eqref{eq:mtch-eq-declsub} to $\bjmtch{\vty}{\ty_1}$
and $\bjmtch{\vty}{\ty_2}$. 
Now it suffices to show that $\bjsub{\vty}{\ty_2}$ follows from
$\bjsub{\vty}{\ty_1}$ and $\bjsub{\ty_1}{\ty_2}$, 
which is trivially true by~\RD{Trans}.
\qed

\begin{lemma}[Completeness of Declarative Subtyping]\label{lem:declsub-complete}
\[	
\forall \ty_1, \ty_2.\ 
(\bjsemsub{\ty_1}{\ty_2} \implies\ \bjsub{\ty_1}{\ty_2})
\]
\end{lemma}

This direction of~\thmref{thm:declsub-correct} is more challenging. 
The key observation here is that \lemref{lem:declsub-complete} can be shown for 
$\ty_1$ of the form $\vty_1 \cup \vty_2 \cup \ldots \cup \vty_n$
(we omit parenthesis because union is associative).
In this case, in the definition of $\bjsemsub{\ty_1}{\ty_2}$ the only
$\vty$s that match $\ty_1$ and $\ty_2$ are $\vty_i$. 
By~\eqref{eq:mtch-eq-declsub} we know that matching implies subtyping,
so we also have $\bjsub{\vty_i}{\ty_2}$.
From the latter, it is easy to show that $\bjsub{\ty_1}{\ty_2}$ because
$\ty_1$ is just a union of value types, 
and subtyping of the left-hand side union amounts to subtyping its components,
according to the \RD{UnionL} rule.

\paragraph{Normal Form}
We say that a type $\ty \equiv \vty_1 \cup \vty_2 \cup \ldots \cup \vty_n$
is in \defemph{normal form} and denote this fact by $\InNF(\ty)$ 
(formal definition of $\InNF$ is given 
in~\figref{fig:bjsem-innf}, \appref{app:nf}).
For each type \ty, there is an equivalent normalized type 
that can be computed with the function $\NF$
defined in~\figref{fig:bjsem-calc-nf} (the auxiliary function $\unprs$ 
can be found in~\figref{fig:bjsem-calc-nf-full}, \appref{app:nf}).
Note that abstract nominal types are unfolded into unions of all their
value subtypes. A pairs gets rewritten into a union of value pairs, 
thus producing a type in the disjunctive normal form.

Using the fact that every type can be normalized,
and that declarative subtyping is complete for normalized types, 
we can finally prove \lemref{lem:declsub-complete}.

\begin{lemma}[Properties of the Normal Form]\label{lem:declsub-nf}
\[
\forall \ty.\ (\InNF(\NF(\ty))\;\ \land\;
\bjsub{\ty}{\NF(\ty)}\;\ \land\; \bjsub{\NF(\ty)}{\ty})
\]
\end{lemma}

\begin{lemma}[Completeness for Normalized Types]
\label{lem:nf-declsub-complete}
\[
	\forall \ty_1, \ty_2 \Alt \InNF(\ty_1).\
	(\bjsemsub{\ty_1}{\ty_2} \implies\ \bjsub{\ty_1}{\ty_2})
\]
\end{lemma}

\begin{lemma}\label{lem:nf-semsub}
$
\forall \ty_1, \ty_2.\ 
(\bjsemsub{\ty_1}{\ty_2} \implies\ \bjsemsub{\NF(\ty_1)}{\ty_2})
$
\end{lemma}

%\begin{figure}
%  \[
%	\begin{array}{rcl}
%	\NF: \Type &\rightarrow& \Type \\
%	\NF(\cname) &=& \cname \\
%	\colorbox{light-gray}{\NF(\tyreal)} &=&
%	\colorbox{light-gray}{\tyunion{\tyint}{\tyflt}} \\
%	\colorbox{light-gray}{\NF(\tynum)} &=&
%	\colorbox{light-gray}{\tyunion{\tyunion{\tyint}{\tyflt}}{\tycmplx}} \\
%	\NF(\typair{\ty_1}{\ty_2}) &=& \unprs(\NF(\ty_1), \, \NF(\ty_2))	\\
%	\NF(\tyunion{\ty_1}{\ty_2}) &=& \tyunion{\NF(\ty_1)}{\NF(\ty_2)} \\
%	\end{array}
%  \]
%	\caption{Computing normal form of \BetaJulia types}
%	\label{fig:bjsem-calc-nf}
%\end{figure}

\noindent
\Proof (Lemma~\ref{lem:declsub-complete}).
We know $\bjsemsub{\ty_1}{\ty_2}$, and we need to show $\bjsub{\ty_1}{\ty_2}$.
First, we apply \lemref{lem:nf-semsub} to $\bjsemsub{\ty_1}{\ty_2}$,
and then \lemref{lem:nf-declsub-complete},
this gives us $\bjsub{\NF(\ty_1)}{\ty_2}$.
Using \lemref{lem:declsub-nf} and \RD{Trans}, we can show
$\bjsub{\ty_1}{\ty_2}$.
\qed

\subsection{Reductive Subtyping}\label{sec:declsub-correct}
%% -----------------------------------------------------------------------------

\begin{theorem}[Correctness of Reductive Subtyping]\label{thm:redsub-correct}
\[
\forall \ty_1, \ty_2.\ (\bjsubr{\ty_1}{\ty_2} \iff\ \bjsub{\ty_1}{\ty_2})
\]
\end{theorem}

It is relatively easy to show by induction on a derivation of
$\bjsubr{\ty_1}{\ty_2}$ that the reductive subtyping is sound:
for each case we build a corresponding derivation of $\bjsub{\ty_1}{\ty_2}$.
Most of the reductive rules have direct declarative counterparts.
In the case of \RR{*Num} and \RR{UnionR*}, 
we need to additionally use transitivity.
Finally, in the case of \RR{NF}, the induction hypothesis gives us
$\bjsub{\NF(\ty_1)}{\ty_2}$, so we can use 
\lemref{lem:declsub-nf} and \RD{Trans} to derive $\bjsub{\ty_1}{\ty_2}$.

The challenging part of the proof is to show completeness. 
For this, we need to prove that 
the reductive definition is \emph{reflexive}, \emph{transitive},
and \emph{distributive}.
To prove transitivity and distributivity, 
we need several auxiliary statements:
\begin{enumerate}
  \item $\bjsubr{\ty}{\ty'}\ \implies\ \bjsubr{\NF(\ty)}{\ty'}$,
  \item $\bjsubr{\NF(\ty_1)}{\NF(\ty_2)}\;\;\ \land\;\; \bjsubr{\NF(\ty_2)}{\ty_3}$\\
     $\implies \bjsubr{\NF(\ty_1)}{\ty_3}$,
  \item $\bjsubr{\NF(\ty)}{\NF(\ty')}\ \implies\ \bjsubr{\ty}{\ty'}$.
\end{enumerate}
Having all the facts, we can prove completeness by induction
on a derivation of $\bjsub{\NF(\ty_1)}{\ty_2}$.
For details, the reader can refer to the full Coq-proof.

\begin{theorem}[Decidability of Reductive Subtyping]\label{thm:redsub-decidable}
\[
\forall \ty_1, \ty_2.\ 
(\bjsubr{\ty_1}{\ty_2}\quad \lor\quad \lnot \bjsubr{\ty_1}{\ty_2})
\]
\end{theorem}
\noindent
The Coq-proof is available.

% By induction on a derivation of $\InNF(\ty_1)$, we can show
% that the reductive subtyping is decidable for $\ty_1 | \InNF(\ty_1)$.
% In this case, $\ty_1$ is either a value type or a union of normalized types.
% If it is a value type, subtyping is equivalent to matching,
% and matching is decidable.
% If $\ty_1$ is a union $\tyunion{\ty_a}{\ty_b}$, 
% we use induction hypothesis for components and the fact that
% subtyping union on the left-hand side
% $\bjsubr{\tyunion{\ty_a}{\ty_b}}{\ty'}$ implies $\bjsubr{\ty_a}{\ty'}$
% and $\bjsubr{\ty_b}{\ty'}$.
% Since $\InNF(\NF(\ty_1))$, we can use the decidability 
% of $\bjsubr{\NF(\ty_1)}{\ty_2}$ to prove decidability
% of $\bjsubr{\ty_1}{\ty_2}$.