Semantic subtyping has been studied primarily in the context of
\emph{statically typed} languages with \emph{structural} typing. 
For example, \citet{bib:Hosoya:2003:XDuce} defined 
a semantic type system for XML that incorporates unions, products,
and recursive types, with a subtyping algorithm based on tree 
automata~\cite{bib:Hosoya:2005:XML}.
\citet{bib:Frisch:2008:sem-sub} presented decidable semantic subtyping
for a language with functions, products, and boolean combinators 
(union, intersection, negation); the decision procedure 
for $\ty_1 <: \ty_2$ is based on checking
the emptiness of $\ty_1 \setminus \ty_2$. 
\citet{bib:Dardha:2013:semsub-oo} adopted semantic subtyping
to objects with structural types, and \citet{bib:Ancona:2016:sem-sub-oo} 
proposed decidable semantic subtyping for mutable records.
Unlike these works, we are interested in applying semantic reasoning
to a \emph{dynamic} language with \emph{nominal} types.
%In particular, we found that semantic subtyping of abstract types (ubiquitous
%in languages with nominal typing) and union types (ubiquitous in type systems
%with semantic subtyping).

Though {multiple dispatch} is more often found in dynamic languages,
there has been research on safe integration of dynamic dispatch into
statically typed languages~\cite{bib:Chambers:1992:Cecil, 
Castagna:1992:COF:141471.141537, bib:Clifton:2000:MultiJava,
Allen:2011:TCM:2076021.2048140,Park:2019:PSM:3302515.3290324}. 
There, subtyping is used for both
static type checking and dynamic method resolution.
In the realm of dynamic languages, \citet{Bezanson2015AbstractionIT} 
employed subtyping for multiple dynamic dispatch in the Julia language.
Julia has a rich language of type annotations 
(including, but not limited to nominal types, tuples, and unions) 
and a complex subtyping relation~\cite{ZappaNardelli:2018:JSR:3288538.3276483}. 
However, it is not clear whether the subtyping relation is decidable 
or even transitive, and transitivity of subtyping is important
for correct implementation of method resolution.
In this paper, while we work with only a subset of Julia types, 
subtyping is transitive and decidable.

Recently, a framework for building transitive, distributive,
and decidable subtyping of union and intersection types was proposed 
by~\citet{Muehlboeck:2018:EUI:3288538.3276482}.
Our language of types does not have intersection types but features
pair types that distribute over unions in a similar fashion.

