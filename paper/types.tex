We base this work on a small language of types \BetaJulia,
presented in~\figref{fig:bjsem-types}.
Types, denoted by $\ty \in \Type$, include pairs, unions, 
and nominal types: \cname denotes \emph{concrete} nominal types
that can be instantiated,
and \aname denotes \emph{abstract} nominal types.

\begin{figure}
	\[
	\begin{array}{rcl@{\qquad}l}
	\ty \in \Type & ::= & & \text{\emph{Types}}
	\\ &\Alt& \typair{\ty_1}{\ty_2}  & \text{covariant pair}
	\\ &\Alt& \tyunion{\ty_1}{\ty_2} & \text{untagged union}
	\\ &\Alt& \cname  & \text{concrete nominal type}
	\\ &\Alt& \aname  & \text{abstract nominal type}
	\\ \\
	\cname & \in &
	\multicolumn{2}{l}{\{ \tyint, \tyflt, \tycmplx, \tystr \}}
	\\ 
	\aname & \in & \multicolumn{2}{l}{\{ \tyreal, \tynum \}}
	\end{array}
	\]
	\begin{tikzpicture}[sibling distance=4em, level distance=2.25em,
	concrete/.style = {shape=rectangle, draw, align=center}]
	\node { Num }
	child { node { Real }
		child { node[concrete] { Int} }
		child { node[concrete] (NF) { Flt} } }
	child { node[concrete, right=2em of NF] (NC) {Cmplx} }
	;
	\node[concrete, right=2em of NC] {Str} ;
	\end{tikzpicture}
	\caption{\BetaJulia: type grammar and nominal hierarchy}
	\label{fig:bjsem-types}
\end{figure}

To simplify the development, we work with a particular hierarchy
of nominal types (presented in~\figref{fig:bjsem-types} as a tree)
instead of a generic class table.
There are four concrete, leaf types (depicted in rectangles)
and two abstract types in the hierarchy. 
Formally, the hierarchy can be represented with a list of declarations
$\bjdeclsub{n_1}{n_2}$ read as ``$n_1$ is a declared subtype of $n_2$''
where $n$ is either $\cname$ or $\aname$.
In the case of \BetaJulia, the hierarchy is defined as follows:
\[
\NomH = [ \bjdeclsub{\tyreal}{\tynum}, 
\bjdeclsub{\tyint}{\tyreal}, \bjdeclsub{\tyflt}{\tyreal},
\bjdeclsub{\tycmplx}{\tynum} ].
\]
Nominal hierarchies should not have cycles, 
and each type can have only one parent.
%In the spirit of the Julia language, 
%concrete types do not have declared subtypes 
%(i.e. concrete types are leaves).

\paragraph{Value Types}
Only types that can be instantiated induce type tags,
and we call these types \defemph{value types}. 
Their formal definition is given in~\figref{fig:bjsem-value-types}:
value type $\vty \in \VType$ is either a concrete nominal type 
or a pair of value types. 
For example, \tyflt, \typair{\tyint}{\tyint},
and \typair{\tystr}{(\typair{\tyint}{\tyint})} are all value types.
Union types, just as abstract nominal types, are not value types.
Even the type \tyunion{\tyint}{\tyint} is not a value type, 
though it describes the same set of values as the value type \tyint.
Note that each value type is a type. %$\VType \subset \Type$, i.e. 

\begin{figure}
	\[
	\begin{array}{rcl@{\qquad}l}
	\vty \in \VType & ::= & & \text{\emph{Value Types}}
	\\ &\Alt& \cname & \text{concrete nominal type}
	\\ &\Alt& \typair{\vty_1}{\vty_2} & \text{pair of value types}
	\end{array}
	\]
	\caption{Value types} % in \BetaJulia
	\label{fig:bjsem-value-types}
\end{figure}


\subsection{Semantic Interpretation of Types}
%% -----------------------------------------------------------------------------

As mentioned in~\secref{sec:intro}, 
we interpret types as sets of type tags (i.e. value types) instead of values,
so we call this semantic interpretation \emph{tag-based}.
Formally, the interpretation is given by the function
$\interpty{\cdot}$ that maps a type $\ty \in \Type$
into a set of value types $s \in \PVType$,
as presented in~\figref{fig:bjsem-interpretation}.

\begin{figure}
	\[
	\begin{array}{rcl}
	\interpty{\cdot}: \Type &\rightarrow& \PVType \\
	\interpty{\cname}  & = & \{\cname\} \\
	\interpty{\tyreal} & = & \{ \tyint, \tyflt \} \\
	\interpty{\tynum} & = & \{ \tyint, \tyflt, \tycmplx \} \\
	\interpty{\typair{\ty_1}{\ty_2}} & = & \{\typair{\vty_1}{\vty_2} 
	\Alt \vty_1 \in \interpty{\ty_1}, \vty_2 \in \interpty{\ty_2}\}\\
	\interpty{\tyunion{\ty_1}{\ty_2}} & = & 
	\interpty{\ty_1} \cup \interpty{\ty_2}
	\end{array}
	\]
	%  	\interpty{\aname} & = & \{\cname\ \Alt \bjnomsub{\cname}{\aname} \} \\
	%  (= \interpty{\tyreal} \cup \{\tycmplx\})
	\caption{Tag-based semantic interpretation of types} %\BetaJulia 
	\label{fig:bjsem-interpretation}
\end{figure}

The interpretation of a type states what values constitute the type:
$\vty \in \interpty{\ty}$ means that values $\nu$ tagged with \vty
(i.e. instances of \vty) belong to $\ty$.
Thus, in \BetaJulia, a \emph{concrete nominal} type \cname is comprised 
only of its direct instances.\footnote{In the general case, the interpretation
of a concrete nominal type would include itself and all its concrete subtypes.}
\emph{Abstract nominal} types cannot be instantiated, 
but we want their interpretation to reflect the nominal hierarchy:
for example, a \tynum value 
is either a concrete complex or a real number, which, in turn,
is either a concrete integer or a floating point value.
Therefore, the set of values of type \tynum 
is described by the set of value types $\{\tycmplx, \tyint, \tyflt\}$.
More generally, the interpretation of an abstract nominal type \aname
can be given as follows: %in the following way:
\[
\interpty{\aname} = \{\cname\ \Alt \bjnomsub{\cname}{\aname} \},
\]
where the relation $\bjnomsub{n_1}{n_2}$ means that 
$n_1$ is transitively a declared subtype of~$n_2$:
%or, in other words, it is a transitive closure of $\bjdeclsub{n_1}{n_2}$:
\begin{mathpar}
	\inferrule*[right=]
	{ \bjdeclsub{n_1}{n_2} \in \NomH }
	{ \bjnomsub{n_1}{n_2} }
	
	\inferrule*[right=]
	{ \bjnomsub{n_1}{n_2} \\ \bjnomsub{n_2}{n_3} }
	{ \bjnomsub{n_1}{n_3} }.
\end{mathpar}
Finally, \emph{pairs} and \emph{unions} are interpreted
set-theoretically.

Once we have the tag interpretation of types, we define 
\defemph{tag-based semantic subtyping} in a usual manner, 
as the subset relation:
\begin{equation}\label{eq:truesemsub-def}
\bjtruesemsub{\ty_1}{\ty_2} \quad \defsign \quad
\interpty{\ty_1} \subseteq \interpty{\ty_2}.
\end{equation}

%and the set of values represented by a type \ty can be characterized
%with the set of value types~$\vty_i$
