A programming language is usually called \defemph{dynamically typed}
if no type checking is performed statically
but errors related to types \emph{are} raised at run-time.
For example, consider the simple code below, assuming that
addition \jlcode{(+)} is defined only on integers, 
string concatenation \jlcode{(\&\&)}~--- only on strings,
and no implicit conversion between types is allowed:
\begin{verbatim}
(x + 5) && "hello"
\end{verbatim}
%We assume that addition \jlcode{(+)} is defined only on integers, 
%string concatenation \jlcode{(\&\&)}~--- only on strings,
%and no implicit conversion between types is allowed.
In a dynamically typed language, two scenarios of running the code are possible,
depending on the run-time value of \jlcode{x}.
(1) If \jlcode{x} contains a non-integer value $v$, 
the program raises a type error "$v$ is not an integer",
because addition expects integers as both arguments.
(2) If \jlcode{x} contains an integer,
evaluation of \jlcode{(x + 5)} succeeds and produces an integer $n$.
After that, the program raises a (different) type error, "$n$ is not a string", 
because concatenation works only for strings.

A statically typed language would reject the program above
because it ``knows'' that the program will definitely fail at run-time.
To know, the type checker needs to reason about \emph{expressions}
(such as \jlcode{x + 5}) and approximate types of values they can produce.
%In this case, it knows that the expression \jlcode{(x + 5)} will produce
%an integer, and thus, concatenation can never succeed.
Note that in the case of dynamic typing, we only care about
types of \emph{values}. %, not arbitrary expressions.
%First, since the result of addition is always an integer, concatenation
%of \jlcode{(x + 5)} and string \jlcode{"hello"} cannot succeed.
%Second, depending on the static type of variable \jlcode{x}, the addition
%\jlcode{(x + 5)} itself is either well-typed (if \jlcode{x} is known to be
%an integer) or ill-typed (if \jlcode{}).

%In some dynamically typed languages, e.g. Julia, 
%In the Julia language, types are used primarily for 
%\emph{multiple dynamic dispatch}


