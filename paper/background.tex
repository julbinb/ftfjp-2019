A programming language is usually called \defemph{dynamically typed}
if no type checking is performed statically
but errors related to types \emph{are} raised at run-time.
For example, consider the simple code below, assuming that
multiplication \jlcode{(*)} is defined only on integers, 
string concatenation \jlcode{(++)}~--- only on strings,
and no implicit conversion between types is allowed.
\begin{verbatim}
    (x * 5) ++ "hello"
\end{verbatim}
%We assume that addition \jlcode{(+)} is defined only on integers, 
%string concatenation \jlcode{(\&\&)}~--- only on strings,
%and no implicit conversion between types is allowed.
In a dynamically typed language, two scenarios of running the code are possible
depending on the run-time value of \jlcode{x}.
(1) If \jlcode{x} contains a non-integer value $v$, 
the program raises a type error "$v$ is not an integer",
because multiplication expects integers for both arguments.
(2) If \jlcode{x} contains an integer,
evaluation of \jlcode{(x * 5)} succeeds and produces an integer $n$,
but then, the program raises a (different) type error, "$n$ is not a string", 
because concatenation expects string arguments.
Note that we only care about %in the case of dynamic typing,
types of \emph{values}. By contrast, in a statically typed language,
the static checker needs to reason about whole expressions and approximate
types of values they might produce.

We can add flexibility to our imaginary dynamically typed language
by the means of \defemph{multiple dynamic dispatch} (MDD), 
which we explain in a bit.
So far, we assumed that multiplication and concatenation have only one 
definition each. For instance, they might be implemented like this 
(we use the Julia syntax for function definitions):
\begin{verbatim}
     *(x::Int, y::Int) = integer_mult(x, y)
    ++(a::Str, b::Str) = string_concat(a, b)
\end{verbatim}

With MDD, we can have more than one definition for the same function, 
with each definition expecting different argument types.


at run-time, each function call will be \emph{dispatched} 
to ``the best'' definition based on the types of the arguments.
A single definition is usually called a \defemph{method}. 
%and the set of all methods
%with the same name is called a \emph{generic function}.
Different methods expect values of different types. % as their arguments.
For instance, let us add a concatenation method\footnote{Function 
	\jlftcode{string} converts an argument to a string.}:
% of an integer and a string
\begin{verbatim}
    ++(x::Int, s::Str) = string(x) ++ s
\end{verbatim}
%Now the generic concatenation function has two methods, 
Now, scenario (2) of running the example changes:
evaluation of \jlcode{(x * 5)} produces an integer $n$, 
and then the method \jlcode{++(::Int, ::Str)} is called on 
\jlcode{(}$n$\jlcode{, "hello")},
which succeeds producing a string.
Note that if we replace \jlcode{"hello"} with an integer, the program will fail
again because neither of the concatenation methods accepts two integers.

Types used in method signatures can be more complex than 
just \jlcode{Int} and \jlcode{Str}. 
In this paper, we work with a subset of Julia types, called \BetaJulia,
presented in~\figref{fig:bjsem-types}.
Types denoted by $\ty \in \Type$ include pairs,
unions, and nominal types in the spirit of Julia:
\cname denotes \emph{concrete} nominal types that can be used to create values,
and \aname denotes \emph{abstract} nominal types.\footnote{
In this terminology, types \jlfttype{Int64} and \jlfttype{Signed} discussed 
in the introduction are a concrete and abstract type, respectively.}

To determine which method to call (if any), the Julia language relies on
\defemph{subtyping}.
Let us define couple more methods:
\begin{verbatim}
*(x::Real, y::Real) = 
    float_mult(float(x), float(y))
*(s::Str,  y::Union{Int,Flt}) = length(s) * y
\end{verbatim}
Now, consider a call:
\begin{verbatim}
"hello" * 3.5
\end{verbatim}
 

How does the run-time know which method to call, if any? A solution adopted
by the Julia language is to use \defemph{subtyping}: 
between a call and a method signature to find applicable methods,
and between method signatures to determine the best method. 
For this, method signatures and function calls are represented as 
tuple types\footnote{Only argument types are taken into account; 
	the return type is ignored.}.
Thus, in our running example, method signatures of \jlcode{**} 
are represented with types
\jlcode{(Str, Str)} and \jlcode{(Int, Str)}, and the call to concatenation
in scenario (2)~--- with \jlcode{(Int, Str)}. 
Naturally, Julia would \emph{dispatch}
the call to the \jlcode{(Int, Str)} method.



%which consists of pairs (instead of n-ary tuples), union types,
%and a number of nominal types.
%The grammar of \BetaJulia is presented in~\figref{fig:bjsem-types}. 


\begin{figure}
	\[
	\begin{array}{rcl@{\qquad}l}
	\ty \in \Type & ::= & & \text{\emph{Types}}
	\\ &\Alt& \typair{\ty_1}{\ty_2}  & \text{covariant pair}
	\\ &\Alt& \tyunion{\ty_1}{\ty_2} & \text{untagged union}
	\\ &\Alt& \cname  & \text{concrete nominal type}
	\\ &\Alt& \aname  & \text{abstract nominal type}
	\\ \\
	\cname & \in &
	\multicolumn{2}{l}{\{ \tyint, \tyflt, \tycmplx, \tystr \}}
	\\ 
	\aname & \in & \multicolumn{2}{l}{\{ \tyreal, \tynum \}}
	\end{array}
	\]
	\begin{tikzpicture}[sibling distance=4em, level distance=2.25em,
	concrete/.style = {shape=rectangle, draw, align=center}]
	\node { Num }
	child { node { Real }
		child { node[concrete] { Int} }
		child { node[concrete] (NF) { Flt} } }
	child { node[concrete, right=2em of NF] (NC) {Cmplx} }
	;
	\node[concrete, right=2em of NC] {Str} ;
	\end{tikzpicture}
	\caption{\BetaJulia: type grammar and nominal hierarchy}
	\label{fig:bjsem-types}
\end{figure}

%A statically typed language would reject the program above
%because it ``suspects'' that the program will fail at run-time.
%Static checking requires reasoning about \emph{expressions}
%(such as \jlcode{x + 5}) and approximating types of values they can produce.
%%In this case, it knows that the expression \jlcode{(x + 5)} will produce
%%an integer, and thus, concatenation can never succeed.
%Note that in the case of dynamic typing, we only care about
%types of \emph{values}. %, not arbitrary expressions.
%First, since the result of addition is always an integer, concatenation
%of \jlcode{(x + 5)} and string \jlcode{"hello"} cannot succeed.
%Second, depending on the static type of variable \jlcode{x}, the addition
%\jlcode{(x + 5)} itself is either well-typed (if \jlcode{x} is known to be
%an integer) or ill-typed (if \jlcode{}).

%In some dynamically typed languages, e.g. Julia, 
%In the Julia language, types are used primarily for 
%\emph{multiple dynamic dispatch}


