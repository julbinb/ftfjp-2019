The declarative subtyping is defined in~\figref{fig:bjsem-decl-sub}.
A big part of the definition is comprised of
the rules that are typically used to define syntactic subtyping 
in languages with base types, unions, and pairs.
Namely, reflexivity and transitivity (\RD{Refl} and \RD{Trans}), 
subtyping of pairs (\RD{Pairs}),
and subtyping of unions (\RD{UnionL}, \RD{UnionR1}, \RD{UnionR2}).
Though \RD{UnionR*} rules are seemingly very strict, 
transitivity allows us to derive judgments such as
$\bjsub{\tyint}{\tyunion{\tystr}{\tyreal}}$ via
$\bjsub{\tyint}{\tyreal}$ and $\bjsub{\tyreal}{\tyunion{\tystr}{\tyreal}}$.
Note that we do need the syntactic definition of subtyping
to be \emph{reflexive} and \emph{transitive}
because so is the semantic subtyping (due to the subset relation).

Semantic subtyping also forces us to add rules 
for distributing pairs over unions, \RD{Distr1} and \RD{Distr2}. 
The reason is that types such as 
\tyunion{(\typair{\tystr}{\tyint})}{(\typair{\tystr}{\tyflt})}
and \typair{\tystr}{(\tyunion{\tyint}{\tyflt})} 
have the same semantic interpretation 
$\{\typair{\tystr}{\tyint}, \typair{\tystr}{\tyflt}\}$.
Therefore, the types need to be equivalent within the declarative definition,
i.e., subtyping between them should hold in both directions.
One direction is trivial:
\begin{mathpar}{\small
\inferrule*[right=]
{ \inferrule*[right=]
  { \bjsub{\tystr}{\tystr} \\ \bjsub{\tyint}{\tyunion{\tyint}{\tyflt}} }
  { \bjsub{\typair{\tystr}{\tyint}}
  	  {\typair{\tystr}{(\tyunion{\tyint}{\tyflt})}} } \\
  \inferrule*[right=]
  { \ldots }
  { \bjsub{\typair{\tystr}{\tyflt}}
  	  {\ldots} } }
{ \bjsub{\tyunion{(\typair{\tystr}{\tyint})}{(\typair{\tystr}{\tyflt})}}
	{\typair{\tystr}{(\tyunion{\tyint}{\tyflt})}} }.
}\end{mathpar}
But the other direction,  
\[
\bjsub{\typair{\tystr}{(\tyunion{\tyint}{\tyflt})}}
  {\tyunion{(\typair{\tystr}{\tyint})}{(\typair{\tystr}{\tyflt})}},
\]
cannot be derived without \RD{Distr2} rule: 
\typair{\tystr}{(\tyunion{\tyint}{\tyflt})} is 
not a subtype of either \typair{\tystr}{\tyint} or \typair{\tystr}{\tyflt},
so we cannot apply \RD{UnionR*} rules\footnote{Transitivity
  does not help in this case.}.

Finally, let us look at subtyping of nominal types.
There are four obvious rules coming directly 
from the nominal hierarchy, for instance, \RD{RealNum} mirrors the fact 
that $\bjdeclsub{\tyreal}{\tynum} \in \NomH$.
Using these rules, judgments such as $\bjsub{\tyint}{\tynum}$ 
can be derived by transitivity.
The most interesting rules in the system are
\RD{RealUnion} and \RD{NumUnion} (\colorbox{light-gray}{highlighted}
in~\figref{fig:bjsem-decl-sub})~--- they are dictated by semantic subtyping
similarly to the distributivity rules.
Namely, we need them to achieve equivalence
between, e.g., types \tyunion{\tyint}{\tyflt} and \tyreal, 
which have the same interpretation $\{\tyint, \tyflt\}$.

%Before we define an inductive declarative relation $\bjsub{\ty_1}{\ty_2}$.

%Besides, the ability to reason about types directly,
%without appealing to all their value subtypes.
%Therefore, we provide  definitions of subtyping
%in the form of inductive rule

\begin{figure}
	\begin{mathpar}
		\inferrule*[right=SD-Refl]
		{ }
		{ \bjsub{\ty}{\ty} }
		
		\inferrule*[right=SD-Trans]
		{ \bjsub{\ty_1}{\ty_2} \\ \bjsub{\ty_2}{\ty_3} }
		{ \bjsub{\ty_1}{\ty_3} }		
		\\
		
		\inferrule[SD-IntReal]
		{ }
		{ \bjsub{\tyint}{\tyreal} }
		
		\inferrule[SD-FltReal]
		{ }
		{ \bjsub{\tyflt}{\tyreal} }
		\\
		
		\inferrule[{SD-RealNum}]
		{ }
		{ \bjsub{\tyreal}{\tynum} }
	
		\inferrule[{SD-CmplxNum}]
		{ }
		{ \bjsub{\tycmplx}{\tynum} }
		\\
		
		\colorbox{light-gray}{$
		\inferrule[SD-RealUnion]
		{ }
		{ \bjsub{\tyreal}{\tyunion{\tyint}{\tyflt}} }
		$}
		
		\colorbox{light-gray}{$
		\inferrule[SD-NumUnion]
		{ }
		{ \bjsub{\tynum}{\tyunion{\tyreal}{\tycmplx}} }
		$}
		\\
		
		\inferrule*[right=SD-Pair]
		{ \bjsub{\ty_1}{\ty'_1} \\ \bjsub{\ty_2}{\ty'_2} }
		{ \bjsub{\typair{\ty_1}{\ty_2}}{\typair{\ty'_1}{\ty'_2}} }
		\\
		
		\inferrule*[right=SD-UnionL]
		{ \bjsub{\ty_1}{\ty'} \\ \bjsub{\ty_2}{\ty'} }
		{ \bjsub{\tyunion{\ty_1}{\ty_2}}{\ty'} }
		\\
		
		\inferrule[{SD-UnionR1}]
		{ }
		{ \bjsub{\ty_1}{\tyunion{\ty_1}{\ty_2}} }
		
		\inferrule[{SD-UnionR2}]
		{ }
		{ \bjsub{\ty_2}{\tyunion{\ty_1}{\ty_2}} }
		\\
		
		\inferrule*[right=SD-Distr1]
		{ }
		{ \bjsub{\typair{(\tyunion{\ty_{11}}{\ty_{12}})}{\ty_2}}
			{\tyunion{(\typair{\ty_{11}}{\ty_2})}{(\typair{\ty_{12}}{\ty_2})}} }
		
		\inferrule*[right=SD-Distr2]
		{ }
		{ \bjsub{\typair{\ty_1}{(\tyunion{\ty_{21}}{\ty_{22}})}}
			{\tyunion{(\typair{\ty_1}{\ty_{21}})}{(\typair{\ty_1}{\ty_{22}})}} }
	\end{mathpar}
	\caption{Declarative Subtyping for \BetaJulia}
	\label{fig:bjsem-decl-sub}
\end{figure}


\subsection{Correctness of Declarative Subtyping}
%% -----------------------------------------------------------------------------

In order to show that the declarative syntactic subtyping is correct
with respect to the semantic subtyping,
we need to prove that the former is sound and complete with respect
to the latter one, that is:
\[
\forall \ty_1, \ty_2.\ (\bjsub{\ty_1}{\ty_2} \iff \bjtruesemsub{\ty_1}{\ty_2}).
\]
As discussed in~\secref{sec:syn-model-of-semsub}, there is 
a sound and complete syntactic model of semantic subtyping,
$\bjsemsub{\ty_1}{\ty_2}$.
So for proofs, we will be actually using the model.
%, based on the straightforward matching relation.
\begin{equation}\label{eq:declsub-eq-semsub}
\forall \ty_1, \ty_2.\ (\bjsub{\ty_1}{\ty_2} \iff \bjsemsub{\ty_1}{\ty_2}).
\end{equation}

\begin{figure}
  \[
	\begin{array}{rcl}
	\NF: \Type &\rightarrow& \Type \\
	\NF(\cname) &=& \cname \\
	\colorbox{light-gray}{\NF(\tyreal)} &=&
	\colorbox{light-gray}{\tyunion{\tyint}{\tyflt}} \\
	\colorbox{light-gray}{\NF(\tynum)} &=&
	\colorbox{light-gray}{\tyunion{\tyunion{\tyint}{\tyflt}}{\tycmplx}} \\
	\NF(\typair{\ty_1}{\ty_2}) &=& \unprs(\NF(\ty_1), \, \NF(\ty_2))	\\
	\NF(\tyunion{\ty_1}{\ty_2}) &=& \tyunion{\NF(\ty_1)}{\NF(\ty_2)} \\
	\end{array}
  \]
	\caption{Computing Normal Form of Types in \BetaJulia}
	\label{fig:bjsem-calc-nf}
\end{figure}
