While the semantic approach does enable intuitive %set-theo\-re\-tic 
reasoning about subtyping,
we also need to be able to build the subtyping algorithm 
to compute subtypes.
However, the semantic definition of subtyping~\eqref{eq:truesemsub-def}
%,\eqref{eq:semsub-def} 
cannot be directly computed because of the quantification $\forall \vty$.
Therefore, we provide a \emph{syntactic} definition, 
equivalent to the semantic one, which is straightforward to implement.

We do this in two steps. 
First, we give an inductive definition, called \emph{declarative}, 
that is handy to reason about; 
we prove it equivalent to the semantic definition.
Second, we provide a \emph{reductive}, syntax-directed definition of subtyping
and prove it equivalent to the declarative one 
(and, hence, the semantic definition as well).
We prove that the reductive subtyping is decidable, 
i.e. for any two types $\ty_1$ and $\ty_2$,
it is possible to prove that either $\ty_1$ is a subtype of $\ty_2$, 
or it is not.
The proofs are mechanized in Coq, and since Coq logic is constructive,
the decidability proof essentially gives a subtyping algorithm.
However, it is also possible to devise an algorithm
as a straightforward recursive function.

\subsection{Declarative Subtyping}
%% -----------------------------------------------------------------------------

The declarative definition of subtyping is provided in~\figref{fig:bjsem-decl-sub}.
The definition is mostly comprised of the standard rules
of syntactic subtyping for unions and pairs:
namely, reflexivity and transitivity (\RD{Refl} and \RD{Trans}), 
subtyping of pairs (\RD{Pairs}),
and subtyping of unions (\RD{UnionL}, \RD{UnionR1}, \RD{UnionR2}).
Though \RD{UnionR*} rules are seemingly very strict 
(they require the left-hand side type to be syntactically equivalent
to a part of the right-hand side type), 
transitivity allows us to derive judgments such as
$\bjsub{\tyint}{(\tyunion{\tystr}{\tyreal})}$ via
$\bjsub{\tyint}{\tyreal}$ and $\bjsub{\tyreal}{\tyunion{\tystr}{\tyreal}}$.
Note that we do need the syntactic definition of subtyping
to be \emph{reflexive} and \emph{transitive} because so is the subset relation,
which is used to define semantic subtyping.

Semantic subtyping also forces us to add rules 
for distributing pairs over unions, \RD{Distr1} and \RD{Distr2}. 
For example, consider two types,
\tyunion{(\typair{\tystr}{\tyint})}{(\typair{\tystr}{\tyflt})}
and \typair{\tystr}{(\tyunion{\tyint}{\tyflt})}.
They have the same semantic interpretation~---
$\{\typair{\tystr}{\tyint}, \typair{\tystr}{\tyflt}\}$~---
so they are equivalent.
Therefore, we should also be able to derive their equivalence
using the declarative definition,
i.e. declarative subtyping should hold in both directions.
One direction is trivial:
\begin{mathpar}{\small
\inferrule*[right=]
{ \inferrule*[right=]
  { \bjsub{\tystr}{\tystr} \\ \bjsub{\tyint}{\tyunion{\tyint}{\tyflt}} }
  { \bjsub{\typair{\tystr}{\tyint}}
  	  {\typair{\tystr}{(\tyunion{\tyint}{\tyflt})}} } \\
  \inferrule*[right=]
  { \ldots }
  { \bjsub{\typair{\tystr}{\tyflt}}
  	  {\ldots} } }
{ \bjsub{\tyunion{(\typair{\tystr}{\tyint})}{(\typair{\tystr}{\tyflt})}}
	{\typair{\tystr}{(\tyunion{\tyint}{\tyflt})}} }.
}\end{mathpar}
But the other direction,  
\[
\bjsub{\typair{\tystr}{(\tyunion{\tyint}{\tyflt})}}
  {\tyunion{(\typair{\tystr}{\tyint})}{(\typair{\tystr}{\tyflt})}},
\]
cannot be derived without \RD{Distr2} rule. 
%\typair{\tystr}{(\tyunion{\tyint}{\tyflt})} is 
%not a subtype of either \typair{\tystr}{\tyint} or \typair{\tystr}{\tyflt},
%so we cannot apply \RD{UnionR*} rules\footnote{Transitivity
%  does not help in this case.}.

The novel part of the definition resides in subtyping of nominal types.
There are four obvious rules coming directly 
from the nominal hierarchy, for instance, \RD{RealNum} mirrors the fact 
that $\bjdeclsub{\tyreal}{\tynum} \in \NomH$.
%Using these rules, judgments such as $\bjsub{\tyint}{\tynum}$ 
%can be derived by transitivity.
But the rules \RD{RealUnion} and \RD{NumUnion}
(\colorbox{light-gray}{highlighted} in~\figref{fig:bjsem-decl-sub})
are new~--- they are dictated by semantic subtyping.
Thus, \RD{RealUnion} allows us to prove the equivalence
of types \tyunion{\tyint}{\tyflt} and \tyreal, 
which are both interpreted as $\{\tyint, \tyflt\}$.

%Before we define an inductive declarative relation $\bjsub{\ty_1}{\ty_2}$.

%Besides, the ability to reason about types directly,
%without appealing to all their value subtypes.
%Therefore, we provide  definitions of subtyping
%in the form of inductive rule

\begin{figure}
	\begin{mathpar}
		\inferrule*[right=SD-Refl]
		{ }
		{ \bjsub{\ty}{\ty} }
		
		\inferrule*[right=SD-Trans]
		{ \bjsub{\ty_1}{\ty_2} \\ \bjsub{\ty_2}{\ty_3} }
		{ \bjsub{\ty_1}{\ty_3} }		
		\\
		
		\inferrule[SD-IntReal]
		{ }
		{ \bjsub{\tyint}{\tyreal} }
		
		\inferrule[SD-FltReal]
		{ }
		{ \bjsub{\tyflt}{\tyreal} }
		\\
		
		\inferrule[{SD-RealNum}]
		{ }
		{ \bjsub{\tyreal}{\tynum} }
	
		\inferrule[{SD-CmplxNum}]
		{ }
		{ \bjsub{\tycmplx}{\tynum} }
		\\
		
		\colorbox{light-gray}{$
		\inferrule[SD-RealUnion]
		{ }
		{ \bjsub{\tyreal}{\tyunion{\tyint}{\tyflt}} }
		$}
		
		\colorbox{light-gray}{$
		\inferrule[SD-NumUnion]
		{ }
		{ \bjsub{\tynum}{\tyunion{\tyreal}{\tycmplx}} }
		$}
		\\
		
		\inferrule*[right=SD-Pair]
		{ \bjsub{\ty_1}{\ty'_1} \\ \bjsub{\ty_2}{\ty'_2} }
		{ \bjsub{\typair{\ty_1}{\ty_2}}{\typair{\ty'_1}{\ty'_2}} }
		\\
		
		\inferrule*[right=SD-UnionL]
		{ \bjsub{\ty_1}{\ty'} \\ \bjsub{\ty_2}{\ty'} }
		{ \bjsub{\tyunion{\ty_1}{\ty_2}}{\ty'} }
		\\
		
		\inferrule[{SD-UnionR1}]
		{ }
		{ \bjsub{\ty_1}{\tyunion{\ty_1}{\ty_2}} }
		
		\inferrule[{SD-UnionR2}]
		{ }
		{ \bjsub{\ty_2}{\tyunion{\ty_1}{\ty_2}} }
		\\
		
		\inferrule*[right=SD-Distr1]
		{ }
		{ \bjsub{\typair{(\tyunion{\ty_{11}}{\ty_{12}})}{\ty_2}}
			{\tyunion{(\typair{\ty_{11}}{\ty_2})}{(\typair{\ty_{12}}{\ty_2})}} }
		
		\inferrule*[right=SD-Distr2]
		{ }
		{ \bjsub{\typair{\ty_1}{(\tyunion{\ty_{21}}{\ty_{22}})}}
			{\tyunion{(\typair{\ty_1}{\ty_{21}})}{(\typair{\ty_1}{\ty_{22}})}} }
	\end{mathpar}
	\caption{Declarative subtyping for \BetaJulia}
	\label{fig:bjsem-decl-sub}
\end{figure}


