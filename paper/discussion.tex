We set out to define semantic subtyping that can be useful in the context
of dynamic languages, however, 
the semantic definition we presented
appears to have an undesired implication for dynamic dispatch.
%it appears that the semantic definition we presented
%has an undesired implication for dynamic dispatch.
In this section, using multiple dispatch as a running example,
we discuss the implication and suggest a solution.
%In this section, we describe in more detail how subtyping
%can be used to implement multiple dynamic dispatch,
%and also discuss implications of using \emph{semantic} subtyping.

Consider the following methods\footnote{In the context of MD, 
	different implementations of 
	the same function are usually called \emph{methods},
	and the set of all methods a \emph{generic function}.}
of the addition function defined in the Julia syntax
(we assume that function \jlcode{flt} converts its argument to a float):
\begin{lstminijl}
+(x::Int, y::Int) = prim_add_int(x, y) 
+(x::Flt, y::Flt) = prim_add_flt(x, y) 
+(x::Int%*$\cup$*)Flt, y::Int%*$\cup$*)Flt) = prim_add_flt(flt(x), ..)
\end{lstminijl}
and the function call \jlcode{3 + 5}.
With multiple dynamic dispatch, the call is resolved at run-time,
based on the types of all arguments. 
But how exactly does method resolution work?
%How exactly does dispatch work?

One approach to implementing multiple dispatch, adopted by some languages
such as Julia~\cite{Bezanson2015AbstractionIT}, 
is to use subtyping on tuple types~\cite{bib:Leavens:1998:mddtuples}.
Namely, method signatures and function calls are interpreted as tuple types,
and then subtyping is used to determine applicable methods 
as well as pick one of them.
In the example above, the three methods are interpreted 
as the following types (from top to bottom):\\
\jlcode{mII $\equiv$ Int $\times$ Int}\\
\jlcode{mFF $\equiv$ Flt $\times$ Flt}\\
\jlcode{mUU $\equiv$ (Int$\cup$Flt) $\times$ (Int$\cup$Flt)}\\
and the call as having type \jlcode{cII $\equiv$ Int $\times$ Int}.
To resolve the call, the language run-time ought to perform two steps.
\begin{enumerate}
  \item Find the applicable methods (or raise an error if there are none). 
    For this, subtyping is checked between
    the type of the call \jlcode{cII} and the method signatures.
    Since \jlcode{cII $<:$ mII} and \jlcode{cII $<:$ mUU} but
    \jlcode{cII $\not{<:}$ mFF}, only two methods are applicable~---
    \jlcode{mII} for integers and \jlcode{mUU} for mixed-type numbers.
  \item Pick the most specific of the applicable methods
    (or raise an error if there is an ambiguity).
    For this, subtyping is checked pairwise between all the applicable methods.
    In this example, naturally, we would like \jlcode{mII} to be called
    for \jlcode{3 + 5}. And indeed, since \jlcode{mII $<:$ mUU}
    and \jlcode{mUU $\not{<:}$ mII},
    the integer addition is picked as the most specific.
\end{enumerate}
As another example, consider the call \jlcode{3.14 + 5}, which type is
\jlcode{Flt $\times$ Int}. There is only one applicable method \jlcode{mUU}
that is a supertype of the call type, so it should be picked.

%It is important to understand 
What happens if the programmer defines
several implementations with the same argument types? 
In the case of a static language, an error can be reported.
In the case of a dynamic language, however, the second implementation
simply replaces the earlier one in the same way as reassignment
to a variable replaces its previous value.

For instance, consider a program 
that contains the three previous implementations of \jlcode{(+)} %from above
and also: % the following:
\begin{lstminijl}
+(x::Real, y::Real) = ...   # mRR
print(3.14 + 5)	 
\end{lstminijl}
According to the semantic subtyping relation, type \jltype{Real} is equivalent
to \jltype{Int$\cup$Flt} in \BetaJulia. 
Therefore, the implementation of \jlcode{mRR} will replace 
\jlcode{mUU} defined earlier, 
and the mixed-type call \jlcode{3.14 + 5} will be dispatched to \jlcode{mRR}.

But there is a problem: %due to the use of semantic subtyping, 
the semantics of the program above will change
%There is a problem with using semantic subtyping for MD:
%the semantics of the program above is not stable.
if the programmer adds a new subtype of \jltype{Real} 
into the nominal hierarchy, e.g. \jlcode{Int8 $<:$ Real}.
In this case, type \jlcode{Real}
stops being equivalent to \jltype{Int$\cup$Flt}
and becomes equivalent to \jltype{Int$\cup$Flt$\cup$Int8}.
Thus, when the program is re-run, type \jlcode{mUU} 
will be a strict subtype of \jlcode{mRR},
so the implementation of \jlcode{mRR} will \emph{not} replace \jlcode{mUU}. 
%Since \jlcode{mUU $<:$ mRR}, 
Therefore, this time, the call \jlcode{3.14 + 5} 
will be dispatched to \jlcode{mUU}, not \jlcode{mRR} as before.

We can gain stability by removing subtyping rules that 
equate abstract nominal types with the union of their subtypes
(i.e. \RD{RealUnion} and \RD{NumUnion} 
in the declarative definition\footnote{To get 
	equivalent reductive subtyping, we need to change 
    the \RR{NF} rule by replacing normalization function $\NF$ with $\NFAt$ 
    (\figref{fig:bjnom-calc-nf-full}, \appref{app:nf}).}
from~\figref{fig:bjsem-decl-sub}).
Then, to fix the discrepancy between the new definition 
and semantic subtyping, the latter should be modified. 
To account for potential extension of the nominal hierarchy,
abstract nominal type \aname can be interpreted 
as containing an extra element $E_\aname$~--- ``future subtype of \aname''.
In the case of \BetaJulia, the new interpretation is as follows:
%``future nominal types'':
%e.g. $\interpty{\tyreal} = \{\tyint, \tyflt, X\}$.
\[
\begin{array}{rcl}
\interpty{\tyreal} & = & \{ \tyint, \tyflt, E_{\tyreal} \} \\
\interpty{\tynum} & = & \{ \tyint, \tyflt, \tycmplx,
                           E_{\tyreal}, E_{\tynum} \}.
\end{array}
\]
%If $E_\aname$ is considered a value type, 
It can be shown that
the modified declarative definition of subtyping
is equivalent to semantic subtyping 
based upon the new interpretation.\footnote{The proof can be found 
	in \texttt{FullAtomicJl} folder of~\cite{bib:MiniJlCoq}.}
