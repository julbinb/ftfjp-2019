\section{Normal Forms}\label{app:nf}
%% -----------------------------------------------------------------------------

\begin{figure}
  \begin{mathpar}
  	\inferrule*[right=NF-ValType]
  	{ }
  	{ \InNF(\vty) }
  	
  	\inferrule*[right=NF-Union]
  	{ \InNF(\ty_1) \\ \InNF(\ty_2) }
  	{ \InNF(\tyunion{\ty_1}{\ty_2}) }
  \end{mathpar}
    \caption{Normal form of types in \BetaJulia}
    \label{fig:bjsem-innf}
\end{figure}

\begin{figure}
  \[
	\begin{array}{rcl}
	\NF: \Type &\rightarrow& \Type \\
	\NF(\cname) &=& \cname \\
	\NF(\tyreal) &=& \tyunion{\tyint}{\tyflt} \\
	\NF(\tynum) &=& \tyunion{\tyunion{\tyint}{\tyflt}}{\tycmplx} \\
	\NF(\typair{\ty_1}{\ty_2}) &=& \unprs(\NF(\ty_1), \, \NF(\ty_2))	\\
	\NF(\tyunion{\ty_1}{\ty_2}) &=& \tyunion{\NF(\ty_1)}{\NF(\ty_2)} \\
	& & \\
	\unprs: \Type\times\Type &\rightarrow& \Type \\
	\unprs(\tyunion{\ty_{11}}{\ty_{12}},\ \ty_2) &=&
	  \tyunion{\unprs(\ty_{11}, \ty_2)}{\unprs(\ty_{12}, \ty_2)} \\
	\unprs(\ty_1,\ \tyunion{\ty_{21}}{\ty_{22}}) &=&
	  \tyunion{\unprs(\ty_1, \ty_{21})}{\unprs(\ty_1, \ty_{22})} \\
	\unprs(\ty_1, \, \ty_2) &=& \typair{\ty_1}{\ty_2}
	\end{array}
  \]
	\caption{Computing normal form of \BetaJulia types}
	\label{fig:bjsem-calc-nf-full}
\end{figure}

\begin{figure}
	\begin{mathpar}
		\inferrule[Atom-CName]
		{ }
		{ \Atom(\cname) }
		
		\inferrule[Atom-AName]
		{ }
		{ \Atom(\aname) }
		\\
		
		\inferrule*[right=NFAt-Atom]
		{ \Atom(\ty) }
		{ \InNFAt(\ty) }
		
		\inferrule*[right=AtNF-Union]
		{ \InNFAt(\ty_1) \\ \InNFAt(\ty_2) }
		{ \InNFAt(\tyunion{\ty_1}{\ty_2}) }
	\end{mathpar}
	\caption{Atomic normal form of types in \BetaJulia}
	\label{fig:bjnom-innf}
\end{figure}

\begin{figure}
	\[
	\begin{array}{rcl}
	\NFAt: \Type &\rightarrow& \Type \\
	\NFAt(\cname) &=& \cname \\
	\NFAt(\aname) &=& \aname \\
	\NFAt(\typair{\ty_1}{\ty_2}) &=& \unprs(\NFAt(\ty_1), \, \NFAt(\ty_2))	\\
	\NFAt(\tyunion{\ty_1}{\ty_2}) &=& \tyunion{\NFAt(\ty_1)}{\NFAt(\ty_2)} \\
	\end{array}
	\]
	\caption{Computing atomic normal form of \BetaJulia types}
	\label{fig:bjnom-calc-nf-full}
\end{figure}

\figref{fig:bjsem-innf} defines the predicate $\InNF(\ty)$, which states
that type $\ty$ is in normal form.
\figref{fig:bjsem-calc-nf-full} contains the full definition of $\NF(\ty)$ 
function, which computes the normal form of a type.

\figref{fig:bjnom-innf} and \figref{fig:bjnom-calc-nf-full} present 
``atomic normal form'', which can be used to define reductive subtyping
that disables derivations such as $\bjsub{\tyreal}{\tyunion{\tyint}{\tyflt}}$.

\section{Non-unique Derivations}\label{app:example-deriv}
%% -----------------------------------------------------------------------------

There are two derivations of
\[\bjsubr{\typair{\tystr}{(\tyunion{\tyint}{\tyflt})}}
         {\typair{\tystr}{\tyreal}}.\]
The shortest derivation:
\begin{mathpar}
\footnotesize
\inferrule*[right=\footnotesize{Pair}]
{ \inferrule[\footnotesize{BaseRefl}]
  { }
  { \bjsubr{\tystr}{\tystr} } \\
  \inferrule*[right=\footnotesize{UnionL}]
  { \inferrule[\footnotesize{IntReal}]
  	{ }{ \bjsubr{\tyint}{\tyreal} } \\
    \inferrule[\footnotesize{FltReal}]
    { }{ \bjsubr{\tyflt}{\tyreal} } }
  { \bjsubr{\tyunion{\tyint}{\tyflt}}{\tyreal} } }
{ \bjsubr{\typair{\tystr}{(\tyunion{\tyint}{\tyflt})}}
	{\typair{\tystr}{\tyreal}} }
\end{mathpar}
The normalization-based derivation:
\begin{mathpar}
\footnotesize
\inferrule*[right=\scriptsize{NF}]
{ \inferrule*[right=\scriptsize{UnionL}]
  { \scriptsize
  	\inferrule
  	{ \inferrule
      { }{ \bjsubr{\tystr}{\tystr} } \\
      \inferrule
      { }{ \bjsubr{\tyint}{\tyreal} } }
  	{ \bjsubr{\typair{\tystr}{\tyint}}{\typair{\tystr}{\tyreal}} } \\
    \scriptsize
    \inferrule
    { \inferrule{ }{\bjsubr{\tystr}{..} } \\
      \inferrule{ }{\bjsubr{\tyflt}{..} } }
    { \bjsubr{\typair{\tystr}{\tyflt}}{\typair{\tystr}{\tyreal}} } }
  { \bjsubr{\tyunion{(\typair{\tystr}{\tyint})}{(\typair{\tystr}{\tyflt})}}
  	  {\typair{\tystr}{\tyreal}} } }
{ \bjsubr{\typair{\tystr}{(\tyunion{\tyint}{\tyflt})}}
	{\typair{\tystr}{\tyreal}} }
\end{mathpar}
%    \inferrule
%	{ \inferrule
%	  { }{ \bjsubr{\tystr}{\tystr} } \\
%	  \inferrule
%	  { }{ \bjsubr{\tyflt}{\tyreal} } }

\section{Overview of Coq Proofs}\label{app:proofs}
%% -----------------------------------------------------------------------------

In this section we give a brief overview of 
the Coq-mecha\-ni\-za\-tion~\cite{bib:MiniJlCoq} of the paper.
When referring to a file \fn{fname}, 
we mean the file \fn{Mechanization/fname} in~\cite{bib:MiniJlCoq}.

\subsection{Definitions}
%% -----------------------------------------------------------------------------

Most of the relevant definitions are in \fn{MiniJl/BaseDefs.v}.
In the table below, 
we show the correspondence between paper definitions (left column)
and Coq definitions (middle column),
possibly with syntactic sugar (right column).

\begin{center}
\begin{tabular}{r|l|l}
\hline
\multicolumn{3}{l}{Types} \\
\hline
\ty  & \coqcode{ty} & \\
\vty & \coqcode{value\_type v} & \\

\hline
\multicolumn{3}{l}{Relations} \\
\hline
$\bjmtch{\vty}{\ty}$ & 
	\coqcode{match\_ty v t} & \coqcode{|- v <\$ t} \\
$\bjsemsub{\ty_1}{\ty_2}$ &
	\coqcode{sem\_sub t1 t2} & \coqcode{||- [t1] <= [t2]} \\
$\bjsub{\ty_1}{\ty_2}$ &
	\coqcode{sub\_d t1 t2} & \coqcode{|- t1 << t2} \\
$\bjsubr{\ty_1}{\ty_2}$ &
	\coqcode{sub\_r t1 t2} & \coqcode{|- t1 << t2} \\
	
\hline
\multicolumn{3}{l}{Auxiliary definitions} \\
\hline
$\InNF(\ty)$ & \coqcode{in\_nf t} & \coqcode{InNF(t)} \\
$\NF(\ty)$   & \coqcode{mk\_nf t} & \coqcode{MkNF(t)} \\
$\unprs(\ty_1, \ty_2)$ &
	\coqcode{unite\_pairs t1 t2} & \\
\hline
\end{tabular}
\end{center}

\subsection{Basic Properties of Normalization Function}
%% -----------------------------------------------------------------------------

File \fn{MiniJl/BaseProps.v} contains several simple properties 
that are needed for proving the major theorems discussed in the paper,
in particular, the following properties of the normalization function $\NF$:

\begin{center}
\begin{tabular}{r|c|l}
\hline
Statement & Ref in text & Name in Coq \\
\hline
$\InNF(\NF(\ty))$ & \eqref{eq:nf-innf} & \coqcode{mk\_nf\_\_in\_nf} \\
$\InNF(\ty) \implies (\NF(\ty) \equiv \ty)$ &
	& \coqcode{mk\_nf\_nf\_\_equal} \\
$\NF(\NF(\ty)) \equiv \NF(\ty)$ &
    & \coqcode{mk\_nf\_\_idempotent} \\
\hline
\end{tabular}
\end{center}

\subsection{Basic Properties of Matching Relation}
%% -----------------------------------------------------------------------------

The following properties are proven in \fn{MiniJl/PropsMatch.v}.

\begin{itemize}
  \item Matching relation is \emph{reflexive}, \coqcode{match\_valty\_\_rflxv}
    (by induction on \vty):
    \[\forall \vty.\ \bjmtch{\vty}{\vty}.\]
  \item The only value type that a value type matches is the value type itself,
    \coqcode{valty\_match\_valty\_\_equal} 
    (by induction on $\bjmtch{\vty_1}{\vty_2}$):
    \[\forall \vty_1, \vty_2.\ (\bjmtch{\vty_1}{\vty_2} \implies 
    \vty_1 \equiv \vty_2).\]
  \item The matching relation is \emph{decidable}, 
    \coqcode{match\_ty\_\_dcdbl}
    (by induction on \vty, then by induction on \ty):
    \[\forall \vty, \ty.\ 
    (\bjmtch{\vty}{\ty}\; \lor\; \lnot [\bjmtch{\vty}{\ty}]).\]
\end{itemize}

\subsection{Correctness of Declarative Subtyping}
%% -----------------------------------------------------------------------------

First, we discuss some auxiliary statements 
that are needed for proving~\thmref{thm:declsub-correct}
(located in \fn{MiniJl/DeclSubProp.v}).

One direction of~\eqref{eq:mtch-eq-declsub},
\begin{equation}\label{eq:mtch-imp-declsub}
\forall \vty, \ty.\ (\bjmtch{\vty}{\ty} \implies\ \bjsub{\vty}{\ty}),
\end{equation}
is proven in \coqcode{match\_ty\_\_sub\_d\_sound} 
by induction on $\bjmtch{\vty}{\ty}$.
The other direction,
\[
\forall \vty, \ty.\ (\bjsub{\vty}{\ty} \implies\ \bjmtch{\vty}{\ty}),
\]
is proven in \coqcode{match\_valty\_\_sub\_d\_complete}
by induction on $\bjsub{\vty}{\ty}$.
The transitivity case, \RD{Trans}, requires a helper statement,
\coqcode{match\_valty\_\_transitive\_on\_sub\_d}:
\begin{equation}\label{eq:mtch-trans}
\forall \ty_1, \ty_2, \vty.\ 
(\bjsub{\ty_1}{\ty_2}\; \land\; \bjmtch{\vty}{\ty_1} \quad \implies \quad
\bjmtch{\vty}{\ty_2}),
\end{equation}
which is proven by induction on $\bjsub{\ty_1}{\ty_2}$.

The equivalence of a type and its normal form~\eqref{eq:nf-declsub-eq}
is shown by induction on \ty
in lemmas \coqcode{mk\_nf\_\_sub\_d1} ($\bjsub{\NF(\ty)}{\ty}$)
and \coqcode{mk\_nf\_\_sub\_d2} ($\bjsub{\ty}{\NF(\ty)}$).

Semantic completeness of declarative subtyping 
for a normalized type~\eqref{eq:nf-declsub-complete},
\[
\forall \ty_1, \ty_2 \Alt \InNF(\ty_1).\
(\bjsemsub{\ty_1}{\ty_2} \implies\ \bjsub{\ty_1}{\ty_2}),
\]
is shown in \coqcode{nf\_sem\_sub\_\_sub\_d} by induction on $\InNF(\ty_1)$.
When $\ty_1 \equiv \vty$, we use~\eqref{eq:mtch-imp-declsub}.
By definition of $\bjsemsub{\vty}{\ty_2}$, we know that 
$\bjmtch{\vty}{\ty_2}$ follows from $\bjmtch{\vty}{\vty}$.\\
When $\ty_1 \equiv \tyunion{\ty_a}{\ty_b}$, 
we use induction hypothesis $\bjsub{\ty_a}{\ty_2}$ and $\bjsub{\ty_b}{\ty_2}$,
\RD{UnionL} rule, and the fact that
\[\forall \vty, \ty_1, \ty_2.\ 
(\bjmtch{\vty}{\ty_i} \implies \bjmtch{\vty}{\tyunion{\ty_1}{\ty_2}}). \]

Finally, soundness and completeness parts of~\thmref{thm:declsub-correct}
(\coqcode{sub\_d\_\_semantic\_sound} and \coqcode{sub\_d\_\_semantic\_complete})
are proven in \fn{MiniJl/Props.v}.
Note that soundness~\eqref{eq:declsub-imp-semsub} is the same as
transitivity of the matching relation~\eqref{eq:mtch-trans}.
The completeness part~\eqref{eq:declsub-complete} is proven as explained
at the end of \secref{sec:declsub-correct}.

\subsection{Correctness of Reductive Subtyping}
%% -----------------------------------------------------------------------------

As discussed in~\secref{sec:redsub-correct}, the soundness part
of \thmref{thm:redsub-correct} 
(lemma \coqcode{sub\_r\_\_sound} in \fn{MiniJl/Props.v}),
\[
\forall \ty_1, \ty_2.\ (\bjsubr{\ty_1}{\ty_2} \implies\ \bjsub{\ty_1}{\ty_2}),
\]
is proven by induction on $\bjsubr{\ty_1}{\ty_2}$.
The only interesting case is the rule \RR{NF} where we have
the induction hypothesis $\bjsub{\NF(\ty_1)}{\ty_2}$ 
and need to show $\bjsub{\ty_1}{\ty_2}$.
Since $\bjsub{\ty_1}{\NF(\ty_1)}$, we can use transitivity (rule \RD{Trans}).

The completeness part of \thmref{thm:redsub-correct}
(lemma \coqcode{sub\_r\_\_complete} in \fn{MiniJl/Props.v}),
\[
\forall \ty_1, \ty_2.\ (\bjsub{\ty_1}{\ty_2} \implies\ \bjsubr{\ty_1}{\ty_2}),
\]
is ultimately proven by induction on $\bjsub{\ty_1}{\ty_2}$.
However, the proof requires showing that reductive subtyping
satisfies the following properties (defined in \fn{MiniJl/RedSubProps.v}):
\begin{itemize}
  \item \emph{Reflexivity}, \coqcode{sub\_r\_\_reflexive} 
    (by induction on \ty):
    \[\forall \ty.\ \bjsubr{\ty}{\ty}. \]
  \item \emph{Transitivity}, \coqcode{sub\_r\_\_transitive}:
    \[
    \forall \ty_1, \ty_2, \ty_3.\ 
    (\bjsubr{\ty_1}{\ty_2}\ \land\ \bjsubr{\ty_2}{\ty_3} \; \implies \;
    \bjsubr{\ty_1}{\ty_3}).
    \]
  \item \emph{Distributivity} of pairs over unions:
    \[
    \bjsubr{\typair{(\tyunion{\ty_{11}}{\ty_{12}})}{\ty_2}}
    {\tyunion{(\typair{\ty_{11}}{\ty_2})}{(\typair{\ty_{12}}{\ty_2})}}
    \]
    and
    \[
    \bjsubr{\typair{\ty_1}{(\tyunion{\ty_{21}}{\ty_{22}})}}
    {\tyunion{(\typair{\ty_1}{\ty_{21}})}{(\typair{\ty_1}{\ty_{22}})}}.
    \]
\end{itemize}

The transitivity proof is done by induction on $\bjsubr{\ty_1}{\ty_2}$.
In some cases it relies on the fact that subtyping a type is the same
as subtyping its normal form,
\begin{equation}\label{eq:redsub-imp-nfredsub1}
\forall \ty.\ (\bjsubr{\ty_1}{\ty_2} \; \iff \; \bjsubr{\NF(\ty_1)}{\ty_2}).
\end{equation}
The right-to-left part follows from \RR{NF}, 
and the left-to-right is shown by induction on $\bjsubr{\ty_1}{\ty_2}$ 
(\coqcode{sub\_r\_\_mk\_nf\_sub\_r1}).
In the \RR{Pair} case of the transitivity proof,
we also need to perform induction on $\bjsubr{\ty_2}{\ty_3}$.
The last case, \RR{NF}, uses the two auxiliary facts:
\[
\forall \ty_1, \ty_2.\ 
(\bjsubr{\ty_1}{\ty_2}\; \implies\; \bjsubr{\NF(\ty_1)}{\NF(\ty_2)}),
\]
proven in \coqcode{sub\_r\_\_mk\_nf\_sub\_r} 
by induction on $\bjsubr{\ty_1}{\ty_2}$
(uses the idempotence of $\NF$),
and $\forall \ty_1, \ty_2, \ty_3.$
\[
\InNF(\ty_1) \land \InNF(\ty_2) \land
(\bjsubr{\ty_1}{\ty_2}) \land (\bjsubr{\ty_2}{\ty_3}) \ \implies \
\bjsubr{\ty_1}{\ty_3},
\]
proven in \coqcode{sub\_r\_nf\_\_transitive} 
by induction on $\bjsubr{\ty_1}{\ty_2}$.

The distributivity proofs use the fact that
\[
\forall \ty_1, \ty_2.\ 
(\bjsubr{\NF(\ty_1)}{\NF(\ty_2)}\; \implies\; \bjsubr{\ty_1}{\ty_2}),
\]
proven in \coqcode{mk\_nf\_sub\_r\_\_sub\_r},
and that normal forms of both types in \RD{Distr*} rules
are in the subtyping relation:
\[
\bjsubr{\NF(\typair{(\tyunion{\ty_{11}}{\ty_{12}})}{\ty_2})}
{\NF(\tyunion{(\typair{\ty_{11}}{\ty_2})}{(\typair{\ty_{12}}{\ty_2})})}
\]
(\coqcode{mk\_nf\_\_distr11}) and
\[
\bjsubr{\NF(\typair{\ty_1}{(\tyunion{\ty_{21}}{\ty_{22}})})}
{\NF(\tyunion{(\typair{\ty_1}{\ty_{21}})}{(\typair{\ty_1}{\ty_{22}})})}
\]
(\coqcode{mk\_nf\_\_distr21}).

\subsection{Decidability of Reductive Subtyping}
%% -----------------------------------------------------------------------------

The proof of \thmref{thm:redsub-decidable},
\[
\forall \ty_1, \ty_2.\ 
(\bjsubr{\ty_1}{\ty_2}\quad \lor\quad \lnot [\bjsubr{\ty_1}{\ty_2}]),
\]
is given by \coqcode{sub\_r\_\_decidable} in \fn{MiniJl/Props.v}.
It relies on the fact (discussed below) 
that reductive subtyping is decidable for $\ty_1$ s.t. $\InNF(\ty_1)$.
\begin{itemize}
  \item Namely, if $\bjsubr{\NF(\ty_1)}{\ty_2}$, 
	then $\bjsubr{\ty_1}{\ty_2}$ by \RR{NF}.
  \item Otherwise, if $\lnot[\bjsubr{\NF(\ty_1)}{\ty_2}]$, 
    which in Coq means
    $\bjsubr{\NF(\ty_1)}{\ty_2} \implies \False$, 
    we can show $\lnot [\bjsubr{\ty_1}{\ty_2}]$ 
    by assuming that $\bjsubr{\ty_1}{\ty_2}$,
    applying~\eqref{eq:redsub-imp-nfredsub1} to it, 
    and thus getting contradiction.
\end{itemize}
Decidability of subtyping of a normalized type,
\[
\forall \ty_1, \ty_2 \Alt \InNF(\ty_1).\ 
(\bjsubr{\ty_1}{\ty_2}\quad \lor\quad \lnot [\bjsubr{\ty_1}{\ty_2}]),
\]
(lemma \coqcode{nf\_sub\_r\_\_decidable} in \fn{MiniJl/RedSubProps.v})
is pro\-ven by induction on $\InNF(\ty_1)$ 
and uses the decidability of the matching relation,
which coincides with reductive subtyping on a value type.
