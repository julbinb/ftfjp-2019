\section{Appendix: Normal Forms}\label{app:nf}
%% -----------------------------------------------------------------------------

\begin{figure}
  \begin{mathpar}
  	\inferrule*[right=NF-ValType]
  	{ }
  	{ \InNF(\vty) }
  	
  	\inferrule*[right=NF-Union]
  	{ \InNF(\ty_1) \\ \InNF(\ty_2) }
  	{ \InNF(\tyunion{\ty_1}{\ty_2}) }
  \end{mathpar}
    \caption{Normal form of types in \BetaJulia}
    \label{fig:bjsem-innf}
\end{figure}

\begin{figure}
  \[
	\begin{array}{rcl}
	\NF: \Type &\rightarrow& \Type \\
	\NF(\cname) &=& \cname \\
	\NF(\tyreal) &=& \tyunion{\tyint}{\tyflt} \\
	\NF(\tynum) &=& \tyunion{\tyunion{\tyint}{\tyflt}}{\tycmplx} \\
	\NF(\typair{\ty_1}{\ty_2}) &=& \unprs(\NF(\ty_1), \, \NF(\ty_2))	\\
	\NF(\tyunion{\ty_1}{\ty_2}) &=& \tyunion{\NF(\ty_1)}{\NF(\ty_2)} \\
	& & \\
	\unprs: \Type\times\Type &\rightarrow& \Type \\
	\unprs(\tyunion{\ty_{11}}{\ty_{12}},\ \ty_2) &=&
	  \tyunion{\unprs(\ty_{11}, \ty_2)}{\unprs(\ty_{12}, \ty_2)} \\
	\unprs(\ty_1,\ \tyunion{\ty_{21}}{\ty_{22}}) &=&
	  \tyunion{\unprs(\ty_1, \ty_{21})}{\unprs(\ty_1, \ty_{22})} \\
	\unprs(\ty_1, \, \ty_2) &=& \typair{\ty_1}{\ty_2}
	\end{array}
  \]
	\caption{Computing normal form of \BetaJulia types}
	\label{fig:bjsem-calc-nf-full}
\end{figure}

\begin{figure}
	\begin{mathpar}
		\inferrule[Atom-CName]
		{ }
		{ \Atom(\cname) }
		
		\inferrule[Atom-AName]
		{ }
		{ \Atom(\aname) }
		\\
		
		\inferrule*[right=NFAt-Atom]
		{ \Atom(\ty) }
		{ \InNFAt(\ty) }
		
		\inferrule*[right=AtNF-Union]
		{ \InNFAt(\ty_1) \\ \InNFAt(\ty_2) }
		{ \InNFAt(\tyunion{\ty_1}{\ty_2}) }
	\end{mathpar}
	\caption{Atomic normal form of types in \BetaJulia}
	\label{fig:bjnom-innf}
\end{figure}

\begin{figure}
	\[
	\begin{array}{rcl}
	\NFAt: \Type &\rightarrow& \Type \\
	\NFAt(\cname) &=& \cname \\
	\NFAt(\aname) &=& \aname \\
	\NFAt(\typair{\ty_1}{\ty_2}) &=& \unprs(\NFAt(\ty_1), \, \NFAt(\ty_2))	\\
	\NFAt(\tyunion{\ty_1}{\ty_2}) &=& \tyunion{\NFAt(\ty_1)}{\NFAt(\ty_2)} \\
	\end{array}
	\]
	\caption{Computing atomic normal form of \BetaJulia types}
	\label{fig:bjnom-calc-nf-full}
\end{figure}

\figref{fig:bjsem-innf} defines the predicate $\InNF(\ty)$, which states
that type $\ty$ is in normal form.
\figref{fig:bjsem-calc-nf-full} contains the full definition of $\NF(\ty)$ 
function, which computes the normal form of a type.

\figref{fig:bjnom-innf} and \figref{fig:bjnom-calc-nf-full} present 
``atomic normal form'', which can be used to define reductive subtyping
that disables derivations such as $\bjsubr{\tyreal}{\tyunion{\tyint}{\tyflt}}$.

%\section{Appendix}
%% -----------------------------------------------------------------------------
%Text of appendix \ldots
